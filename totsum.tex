\documentclass[12pt]{article}
\renewcommand*\rmdefault{ppl}
\usepackage[margin=2cm]{geometry} \geometry{letterpaper}
\usepackage[parfill]{parskip}
\usepackage[utf8]{inputenc}
\usepackage[english]{babel}
\usepackage[style=numeric,backref=true,backrefstyle=none,abbreviate=false,urldate=iso,seconds=true]{biblatex} \addbibresource{totsum.bib}
\usepackage{csquotes}
\usepackage{amssymb,amsmath,amsthm}
\usepackage{graphicx}
\usepackage{tikz}
\usepackage{tikzscale}
\usepackage{pgfplots} \pgfplotsset{compat=1.15}
\usepackage{url}
\usepackage{multicol}
\usepackage[yyyymmdd,hhmmss]{datetime}
\usepackage[bottom]{footmisc}
\usepackage{hyperref}
\usepackage{color} \newcommand{\todo}[0]{{\textcolor{red}{\textbf{TODO: }}}}
\usepackage{enumitem}
\usepackage{listings}
\usepackage{xcolor}
\usepackage[linesnumbered,vlined,boxed,ruled,algonl]{algorithm2e}
\usepackage{orcidlink}
\DeclareGraphicsRule{.tif}{png}{.png}{`convert #1 `dirname #1`/`basename #1 .tif`.png}
\newcommand{\eqn}[1]{\begin{displaymath} #1 \end{displaymath}}
\newcommand{\neqn}[1]{\begin{equation} #1 \end{equation}}
\newcommand{\tweqn}[1]{\begin{displaymath}\resizebox{\textwidth}{!}{$\displaystyle #1 $}\end{displaymath}}
\newcommand{\ntweqn}[1]{\begin{equation}\resizebox{\textwidth}{!}{$\displaystyle #1 $}\end{equation}}
\newcommand{\floor}[1]{{\left\lfloor #1 \right\rfloor}}
\newcommand{\ceil}[1]{{\left\lceil #1 \right\rceil}}
\newcommand{\vect}[1]{\left\langle #1 \right\rangle}
\newcommand{\deriv}[2]{\frac{d #1}{d #2}}
\newcommand{\derop}[1]{\frac{d}{d #1}}
\newcommand{\partiald}[2]{\frac{\partial #1}{\partial #2}}
\newcommand{\partialop}[1]{\frac{\partial}{\partial #1}}
\newcommand{\integral}[4]{\displaystyle\int_{#3}^{#4} \! #1 \, d#2}
\newcommand{\dintegral}[4]{\displaystyle\int\!\!\!\!\int_{#3}^{#4} #1 \, d#2}
\newcommand{\tintegral}[4]{\displaystyle\int\!\!\!\!\int\!\!\!\!\int_{#3}^{#4} #1 \, d#2}
\newcommand{\disp}[0]{\displaystyle}
\newcommand{\abs}[1]{\left\vert #1 \right\vert}
\newcommand{\grad}[0]{\vec{\nabla\!}\,}
\newcommand{\epsil}[0]{\varepsilon}
\newcommand{\eval}[3]{\left. #1 \right|_{#2}^{#3}}
\newcommand{\realpart}[1]{{\mathfrak{Re}\!\left\{#1\right\}}}
\newcommand{\ipart}[1]{{\mathfrak{Im}\!\left\{#1\right\}}}
\newcommand{\set}[1]{{\left\{#1\right\}}}
\newcommand{\lcm}[0]{\operatorname{lcm}}
\newcommand{\defeq}[0]{\overset{\mathrm{def}}{=}}
\newcommand{\oeis}[0]{$^{\texttt{OE}}_{\texttt{IS}}$}
\newcommand{\oeisref}[1]{$^{\texttt{OE}}_{\texttt{IS}}$~\href{https://oeis.org/#1}{#1}}
\newcommand{\quadtext}[1]{\quad \text{#1} \quad}
\newcommand{\qquadtext}[1]{\qquad \text{#1} \qquad}

\allowdisplaybreaks[4]

\def\grabtimezone #1#2#3#4#5#6#7#8#9{\grabtimezoneB}
\def\grabtimezoneB #1#2#3#4#5#6#7{\grabtimezoneC}
\def\grabtimezoneC #1#2'#3'{$#1$#2:#3}
\newcommand{\timezone}[0]{UTC\expandafter \grabtimezone\pdfcreationdate}
\newcommand{\currentdate}{\the\year--\twodigit{\the\month}--\twodigit{\the\day}}
\newcommand{\currentdatetime}{\currentdate\ / \currenttime}%\ \timezone}

\title{Computation of the Totient Summatory Function}
\author{Lucas Augustus Brown \orcidlink{0000-0002-6000-3735}}
\date{\currentdatetime}

\usepackage{fancyhdr}
\usepackage{lastpage}
\pagestyle{fancy}
\fancyhf{}
\lhead{} \chead{Computation of the Totient Summatory Function} \rhead{}
\lfoot{\currentdatetime} \cfoot{} \rfoot{Page \thepage\ of \pageref{LastPage}}
\renewcommand{\headrulewidth}{1pt}
\renewcommand{\footrulewidth}{1pt}
\setlength{\headheight}{15pt}

\begingroup
    \makeatletter
    \@for\theoremstyle:=definition,remark,plain\do{%
        \expandafter\g@addto@macro\csname th@\theoremstyle\endcsname{%
            \addtolength\thm@preskip\parskip
            }%
        }
\endgroup
\makeatletter
\renewenvironment{proof}[1][\proofname]{\par
  \vspace{-\topsep}% remove the space after the theorem
  \pushQED{\qed}%
  \normalfont
  \topsep0pt \partopsep0pt % no space before
  \trivlist
  \item[\hskip\labelsep
        \itshape
    #1\@addpunct{.}]\ignorespaces
}{%
  \popQED\endtrivlist\@endpefalse
  \addvspace{0pt} % some space after
}
\makeatother
\makeatletter
\newenvironment{solution}[1][\proofname]{\par
  \vspace{-\topsep}% remove the space after the theorem
  \pushQED{\qed}%
  \normalfont
  \topsep0pt \partopsep0pt % no space before
  \trivlist
  \item[\hskip\labelsep
        \bfseries
    Solution #1\@addpunct{.}]\ignorespaces
}{%
  \popQED\endtrivlist\@endpefalse
  \addvspace{0pt} % some space after
}
\makeatother
\usepackage{thmtools}
\declaretheorem[style=plain]{theorem}
\declaretheorem[sibling=theorem,style=plain]{corollary}
\declaretheorem[sibling=theorem,style=plain]{lemma}
\declaretheorem[sibling=theorem,style=plain]{proposition}
\declaretheorem[sibling=theorem,style=plain]{conjecture}
\declaretheorem[sibling=theorem,style=definition,qed=$\clubsuit$]{definition}
\declaretheorem[sibling=theorem,style=definition,qed=$\clubsuit$]{observation}
\declaretheorem[sibling=theorem,style=definition,qed=$\clubsuit$]{fact}
\declaretheorem[sibling=theorem,style=definition,qed=$\spadesuit$]{example}
\declaretheorem[sibling=theorem,style=definition]{notation}
\declaretheorem[sibling=theorem,style=definition]{question}
\declaretheorem[sibling=theorem,style=remark]{remark}
\declaretheorem[style=definition]{problem}

\renewcommand\qedsymbol{$\blacksquare$}

\hypersetup{
    pdftitle={On the Summation of the Totient Function},
    pdfauthor={Lucas A. Brown},
    pdfsubject={},
    pdfkeywords={},
    colorlinks=true,
    linkcolor=blue,
    urlcolor=blue,
    citecolor=blue,
    %hidelinks,
}

\definecolor{codegreen}{rgb}{0,0.6,0}
\definecolor{codegray}{rgb}{0.5,0.5,0.5}
\definecolor{codepurple}{rgb}{0.58,0,0.82}
\definecolor{backcolour}{rgb}{0.95,0.95,0.92}

\lstdefinestyle{mystyle}{
    backgroundcolor=\color{backcolour},   
    commentstyle=\color{codegreen},
    keywordstyle=\color{magenta},
    numberstyle=\tiny\color{codegray},
    stringstyle=\color{codepurple},
    basicstyle=\ttfamily\footnotesize,
    breakatwhitespace=false,         
    breaklines=true,                 
    captionpos=b,                    
    keepspaces=true,                 
    numbers=left,                    
    numbersep=5pt,                  
    showspaces=false,                
    showstringspaces=false,
    showtabs=false,                  
    tabsize=2
}

\lstset{style=mystyle}
\newcommand{\showcode}[1]{Filename: \texttt{#1.py} \lstinputlisting[language=Python]{code/#1.py}}
\newcommand{\floordiv}[2]{\floor{\frac{#1}{#2}}}
\newcommand{\dfloordiv}[2]{\floor{\dfrac{#1}{#2}}}
\newcommand{\isqrt}[1]{\floor{\sqrt{#1}}}
\newcommand{\softO}[0]{\widetilde{O}}
\newcommand{\softTheta}[0]{\widetilde{\Theta}}
\SetKw{KwBreak}{break}
\IncMargin{2em}

\begin{document}
\maketitle %\thispagestyle{fancy}

\begin{abstract}
Let $\Phi(n) = \phi(1) + \cdots + \phi(n)$ be the totient summatory function.  We devise an algorithm for computing $\Phi(n)$ in time $\softO(n^{2/3})$ and space $\softO(n^{1/3})$.
\end{abstract}

\section{Notation}

Euler's totient function is denoted $\phi$.  Its summatory function is denoted $\Phi$.

The Dirichlet convolution of $f$ and $g$ is denoted by $f*g$.

The letter $\mu$ is used for both the M\"{o}bius function and an array such that $\mu_k = \mu(k)$.

The letter $M$ is used for both the Mertens function and an array such that $M_k = M(k)$.

We use $\delta(n) = \floor{1/n}$.  Note that this is the identity function for Dirichlet convolution.

\section{Existing algorithms}

The sieve of Eratosthenes can be modified to compute $\phi(n)$ for all $1 \leq n \leq n$, and therefore $\Phi(n)$, in time $\softO(n)$ and space $\softO(n^{1/2})$.\footnote{See file \texttt{totientsumA.py} for an implementation.}  Helfgott's sieve \cite{Helfgott2020} can be used to reduce the memory usage to $\softO(n^{1/3})$ in exchange for a logarithmic time penalty.  (\todo implement that)

Hirsch, Kessler, and Mendlovic outline \cite[\S5.6]{HKM2024} a method to compute $\Phi(n)$ in $\softO(n^{1/2})$ time and $\softO(n^{1/2})$ space.  However, this has never been implemented, and the hidden constants are expected to make the algorithm non-competitive for practical values of $n$.  \todo the HKM prime-counting algorithm has space-time tradeoffs.  What tradeoffs are available for $\Phi$?

\subsection{The Mertens-first algorithm}

By applying the Dirichlet hyperbola method to the convolution $\phi = \mu * I$, where $I(x)=x$, and letting $ab=n$, we obtain the formula
\eqn{\Phi(n) = \sum_{x=1}^{a}\sum_{y=1}^{n/x} \mu(x) \, I(y) + \sum_{y=1}^{b}\sum_{x=1}^{n/y} \mu(x) \, I(y) - \sum_{x=1}^{a}\sum_{y=1}^{b} \mu(x) \, I(y)}
\eqn{ = \sum_{x=1}^{a}\sum_{y=1}^{n/x} y \cdot \mu(x) + \sum_{y=1}^{b}\sum_{x=1}^{n/y} y \cdot \mu(x) - \sum_{x=1}^{a}\sum_{y=1}^{b} y \cdot \mu(x)}
\neqn{\Phi(n) = \underbrace{\sum_{x=1}^{a} \mu(x) \cdot \frac{\floordiv{n}{x} \cdot \left(\floordiv{n}{x} + 1\right)}{2}}_{X} + \underbrace{\sum_{y=1}^{b} y \cdot M(n/y)}_{Y} - \underbrace{\frac{b \cdot (b+1)}{2} \cdot M(a)}_{Z} \label{PhiFormula}}
The labels $X$, $Y$, and $Z$ will be used later.

Suppose that we have an algorithm that can compute $M(x)$ in time $\softO(x^c)$, and note that $c < 1$ is available.  Using a sieve to compute the necessary M\"obius values, but otherwise evaluating this formula na\"{i}vely, takes time
\eqn{\softO\left( a + \sum_{x=1}^b \left(\frac{n}{x}\right)^c + a^c \right)}
\eqn{=\softO\left( a + n^c \integral{x^{-c}}{x}{1}{b} + a^c \right)}
\eqn{=\softO\left( a + n^c\frac{b^{1-c}}{1-c} - n^c\frac{1^{1-c}}{1-c} + a^c \right)}
\eqn{=\softO\left( a + n^c b^{1-c} - n^c + a^c \right)}
\eqn{=\softO\left( a + n^c n^{1-c} a^{c-1} - n^c + a^c \right)}
\eqn{=\softO\left( a + n a^{c-1} - n^c + a^c \right)}
The third term is always dominated by the second, and the fourth is always dominated by the first.
\eqn{=\softO\left( a + n a^{c-1} \right)}
To balance the contributions of the two terms, we take $a = \softO(n^{1/(2-c)})$.

The Del\'{e}glise-Rivat algorithm \cite{DR1996} allows $c=2/3$, and so using it in this algorithm sets $a=3/4$.  The time complexity is then $\softO(n^{3/4})$. 
 The Del\'{e}glise-Rivat algorithm is invoked for arguments up to $n$, so its contribution to the memory usage is $\softO(n^{1/3})$.  If the M\"{o}bius sieving is done with Helfgott's algorithm \cite{Helfgott2020}, then the sieving consumes $\softO(n^{1/4})$ space; if the traditional square-root segmentation is used, then the sieving consumes $\softO(n^{3/8})$ space.\footnote{See file \texttt{totientsumB.py} for an implementation.}

The Mertens function can also be computed with the Helfgott-Thompson algorithm, which takes $\softO(n^{3/5})$ time and $\softO(n^{3/10})$ space.  Evaluating (\ref{PhiFormula}) as described then takes $\softO(n^{5/7})$ time, and we have $a = \softO(n^{5/7})$.  The space usage is then $\softO(n^{3/10})$ inside the Helfgott-Thompson algorithm and either $\softO(n^{5/14})$ or $\softO(n^{5/21})$, depending on which sieving method is used.

This algorithm suffers from the fact that all those $M(n/y)$-values are computed one-at-a-time and are not given a chance to contribute to each other.  This can be ameliorated by another application of the Dirichlet hyperbola method.  This time, we use $\delta = \mu * 1$ and set $\alpha\beta=n$ to obtain
\eqn{\sum_{k=1}^n \delta(k) = \sum_{x=1}^{\alpha}\sum_{y=1}^{n/x} \mu(x) \cdot 1 + \sum_{y=1}^{\beta}\sum_{x=1}^{n/y} \mu(x) \cdot 1 - \sum_{x=1}^{\alpha}\sum_{y=1}^{\beta} \mu(x) \cdot 1}
\eqn{1 = \sum_{x=1}^{\alpha} \mu(x) \floordiv{n}{x} + \sum_{y=1}^{\beta} M(n/y) - M(\alpha) \floor{\beta}}
\neqn{M(n) = 1 + \floor{\beta} M(\alpha) - \sum_{x=1}^{\alpha} \mu(x) \floordiv{n}{x} - \sum_{y=2}^{\beta} M(n/y) \label{MertensRecursion}}
When evaluating (\ref{PhiFormula}), we need to find $\mu(k)$ for $1 \leq k \leq a$, $M(n/k)$ for $1 \leq k \leq b$, and $M(a)$.

When evaluating (\ref{MertensRecursion}), we need to find $\mu(k)$ for $1 \leq k \leq \alpha$, $M(n/k)$ for $2 \leq k \leq \beta$, and $M(\alpha)$.

Clearly, these work well together: we can take $\alpha=a$ (and therefore $\beta=b$), sieve $\mu$ up to $a$, accumulate the values along the way to compute $M$ up to $a$, use (\ref{MertensRecursion}) to compute the remaining Mertens values, and then feed all that data into (\ref{PhiFormula}) to compute $\Phi(n)$.  This results in Algorithm \ref{Algo1}, which I call the \emph{Mertens-first algorithm}.

\begin{algorithm}[H] \label{Algo1}
\DontPrintSemicolon
\caption{Compute $\Phi(n)$ in $\softTheta(n^{2/3})$ time and $\softTheta(n^{1/2})$ space using the Mertens-first algorithm \cite{griff2023}.  See file \texttt{totientsumC.py} for an implementation.}
\KwData{$n \geq 1$}
\KwResult{$\Phi(n)$}
\Begin{
    $a \gets \floor{\softTheta(n^{2/3})}$; $b \gets \floor{n/a}$; $X \gets 0$; $Y \gets 0$; $Z \gets 0$; $m \gets 0$; $s \gets \isqrt{n}$ \label{1-p1start}
    
    \lIf{$\isqrt{n} = \floor{n/\isqrt{n}}$}{$s \gets s-1$}
    
    $\chi \gets \floor{n/s}$
    
    Prepare a segmented Sieve of Eratosthenes to compute $\mu(k)$ for $1 \leq k \leq a$.
    
    Let $\mu$ and $M$ be arrays indexed from $1$ through $\isqrt{n}$, inclusive.
    
    Let $M^\prime$ be an array indexed from $1$ through $\floor{n/\isqrt{n}}$, inclusive, initialized to all zeros. \label{1-endinit}
    \begin{multicols}{2}
    \For{$x=1$ \KwTo $a$}{ \label{1-p12loopstart}
        $v \gets \floor{n/x}$
        
        $m \gets m + \mu(x)$
        
        $X \gets X + \mu(x) \cdot \dfrac{v \cdot (v+1)}{2}$ \label{1-14}
        
        \uIf{$x \leq \isqrt{n}$}{ \label{1-p1a}
            $M_x \gets m$
            
            $\mu_x \gets \mu(x)$ \label{1-p1b}
        }
        \ElseIf{$x = \chi$}{ \label{1-p2a}
        
            \lIf{$v \neq b$}{$M^\prime_{v} \gets m$} \label{1-17}
            
            $s \gets s-1$
            
            $\chi \gets \floor{n/s}$
        }
        \lIf{$x = a$}{$Z \gets m \cdot \dfrac{b \cdot (b+1)}{2}$} \label{1-p2b}
    }
    
    \label{1-p1end}
    
    \emph{lines \ref{1-p3start}--\ref{1-p3end} here}
    
    \columnbreak
    
    \For{$y=b$ \KwTo $1$}{ \label{1-p3start}
        $v \gets \floor{n/y}$
        
        $m \gets 1 - v + \isqrt{v} \cdot M_{\isqrt{v}}$
        
        \For{$x=2$ \KwTo $\isqrt{v}$}{
            $m \gets m - \mu_x \cdot \floor{v/x}$ \label{1-p3mobius}
            
            \uIf{$\floor{v/x} \leq \isqrt{n}$}{
                $m \gets m - M_{\floor{v/x}}$
            }
            \Else{
                $m \gets m - M^\prime_{\floor{n/\floor{v/x}}}$
            }
        }
        $M^\prime_y \gets M^\prime_y + m$
        
        $Y \gets Y + y \cdot M^\prime_y$ \label{1-p3end}
    }
    \end{multicols}
    
    \KwRet $X + Y - Z$
}
\end{algorithm}

The variables $X$, $Y$, and $Z$ in this algorithm correspond to the labels $X$, $Y$, and $Z$ in (\ref{PhiFormula}).

Algorithm \ref{Algo1} has four phases:
\begin{enumerate} \addtocounter{enumi}{-1}
\item In the zeroth phase, lines \ref{1-p1start}--\ref{1-endinit} initialize the computation.
\item In the first phase, we sieve up to $\isqrt{n}$.  This is covered in lines \ref{1-p12loopstart}--\ref{1-p1b}.  In this phase, we sieve the M\"{o}bius function up to $\isqrt{n}$, accumulate its values to compute the Mertens function, save both $\mu$ and $M$, and accumulate terms from part $X$ of (\ref{PhiFormula}).
\item In the second phase, we continue the sieve up to $a$.  This is covered in lines \ref{1-p12loopstart}--\ref{1-p2b}, skipping lines \ref{1-p1a}--\ref{1-p1b}.  In this phase, we continue to accumulate M\"{o}bius values to compute Mertens values, and we continue to accumulate terms from part $X$ of (\ref{PhiFormula}), but we do not save any $\mu$, and only some Mertens values are saved.  As the final act of phase 2, we compute part $Z$ of (\ref{PhiFormula}).  At this point, $X$ and $Z$ are fully evaluated, and nothing has been done about $Y$.
\item In the third phase, lines \ref{1-p3start} through \ref{1-p3end} feed the stored M\"{o}bius and Mertens values into (\ref{MertensRecursion}) to compute the remaining Mertens values in order of increasing argument---that is, we first compute $M(n/b)$, then $M(n/(b-1))$, then ..., and finally $M(n)$.  As each Mertens value is computed, a term from part $Y$ of (\ref{PhiFormula}) becomes available, and we evaluate it accordingly.
\end{enumerate}
Once the third phase is done, $\Phi(n)$ is computed as $X+Y-Z$.

Line \ref{1-17} is gatekept by the condition $v \neq b$.  This is needed to mitigate an overlap in the phases that occurs for some $(a,n)$ pairs.  In such cases, without the gatekeeping, line \ref{1-17} would set $M^\prime_b$ to $M(a)$, which should be its final value, but it then gets modifed in the first iteration through phase 3, which throws things off.  With the condition $v \neq b$ in place, $M^\prime_b$ is not touched until phase 3.

Algorithm \ref{Algo1} takes $\softTheta(n^{2/3})$ time: phases 0--2 combined clearly take $\softTheta(a)$ time, and phase 3 takes time
\eqn{\softTheta \left( \sum_{y=1}^b \left( \isqrt{\frac{n}{y}} - 1 \right) \right)}
%\eqn{= \softTheta \left( \sum_{y=1}^b \left( \isqrt{\frac{n}{y}} \right) - b \right)}
\eqn{= \softTheta \left( \integral{ \sqrt{\frac{n}{y}} }{y}{1}{b} - b \right)}
%\eqn{= \softTheta \left( 2 \sqrt{n} \eval{\sqrt{y}}{y=1}{b} - b \right)}
\eqn{= \softTheta \left( 2 \sqrt{n} \left( \sqrt{b} - 1 \right) - b \right)}
\eqn{= \softTheta \left( \frac{n}{\sqrt{a}} \right).}

Algorithm \ref{Algo1} takes $\softTheta(\sqrt{n})$ space: we use three arrays of $\Theta(\sqrt{n})$ elements each to store the M\"{o}bius and Mertens values, the M\"{o}bius sieving consumes $\softO(\sqrt{a})$ space, and everything else fits in $O(1)$ space.

\section{The Mertens-first algorithm in less space}

We now reduce Algorithm \ref{Algo1}'s memory usage from $\softTheta(\sqrt{n})$ to $\softTheta(\sqrt[3]{n})$.  The first step is to observe that we can move line \ref{1-p3mobius} into phase 1.  The work done in that line is essentially as follows:

\begin{algorithm}[H] \label{Algo1mu}
\DontPrintSemicolon
\caption{An extract from Algorithm \ref{Algo1}}
\Begin{
    \For{$y=b$ \KwTo $1$}{
        \For{$x=2$ \KwTo $\isqrt{n/y}$}{
            $M^\prime_y \gets M^\prime_y - \mu_x \cdot \dfloordiv{n}{yx}$
        }
    }
}
\end{algorithm}

If we can swap the order of the loops, then we will be able to integrate this line into phase 1 and not have to store the M\"{o}bius array.

This extract iterates over all pairs $(y,x)$ such that $1 \leq y \leq b$ and $2 \leq x \leq \sqrt{n/y}$.  The range accessed by $x$ is therefore $2 \leq x \leq \sqrt{n}$, and for each $x$, $y$ ranges over $1 \leq y \leq \min(b, n/x^2)$.  This extract is therefore equivalent to

\begin{algorithm}[H] \label{Algo1mu_redone}
\DontPrintSemicolon
\caption{Algorithm \ref{Algo1mu}, reordered}
\Begin{
    \For{$x=2$ \KwTo $\isqrt{n}$}{
        \For{$y=1$ \KwTo $\min(b,\floor{n/x^2})$}{
            $M^\prime_y \gets M^\prime_y - \mu(x) \cdot \dfloordiv{n}{yx}$
        }
    }
}
\end{algorithm}

Applying this edit to Algorithm \ref{Algo1} yields Algorithm \ref{Algo4}: line \ref{1-p1b} has been replaced with lines \ref{4-15}--\ref{4-17}, and line \ref{1-p3mobius} has been removed entirely.

\begin{algorithm}[H] \label{Algo4}
\DontPrintSemicolon
\caption{Compute $\Phi(n)$ in $\softTheta(n^{2/3})$ time and $\softTheta(n^{1/2})$ space.  See file \texttt{totientsumD.py} for an implementation.}
\KwData{$n \geq 1$}
\KwResult{$\Phi(n)$}
\Begin{
    $a \gets \floor{\softTheta(n^{2/3})}$; $b \gets \floor{n/a}$; $X \gets 0$; $Y \gets 0$; $Z \gets 0$; $m \gets 0$; $s \gets \isqrt{n}$ \label{4-2}
    
    \lIf{$\isqrt{n} = \floor{n/\isqrt{n}}$}{$s \gets s-1$}
    
    $\chi \gets \floor{n/s}$ \label{4-5}
    
    Prepare a segmented Sieve of Eratosthenes to compute $\mu(x)$ for $1 \leq x \leq a$.
    
    Let $M$ be an array indexed from $1$ through $\isqrt{n}$, inclusive.
    
    Let $M^\prime$ be an array indexed from $1$ through $\floor{n/\isqrt{n}}$, inclusive, initialized to all zeros.
    \begin{multicols}{2}
    \For{$x=1$ \KwTo $a$}{
        $v \gets \floor{n/x}$
        
        $m \gets m + \mu(x)$
        
        $X \gets X + \mu(x) \cdot \dfrac{v \cdot (v+1)}{2}$
        
        \uIf{$x \leq \isqrt{n}$}{
            $M_x \gets m$
            
            \If{$x > 1$}{ \label{4-15}
                \For{$y=1$ \KwTo $\min(b,\floor{n/x^2})$}{
                    $M^\prime_y \gets M^\prime_y - \mu(x) \cdot \dfloordiv{n}{yx}$ \label{4-17}
                }
            }
        }
        \ElseIf{$x = \chi$}{
        
            \lIf{$v \neq b$}{$M^\prime_{v} \gets m$}
            
            $s \gets s-1$
            
            $\chi \gets \floor{n/s}$
        }
        \lIf{$x = a$}{$Z \gets m \cdot \dfrac{b \cdot (b+1)}{2}$}
    }
    
    \emph{lines \ref{4-26}--\ref{4-35} here}
    
    \columnbreak
    
    \For{$y=b$ \KwTo $1$}{ \label{4-26}
        $v \gets \floor{n/y}$
        
        $m \gets 1 - v + \isqrt{v} \cdot M_{\isqrt{v}}$ \label{4-27}
        
        \For{$x=2$ \KwTo $\isqrt{v}$}{
            \uIf{$\floor{v/x} \leq \isqrt{n}$}{
                $m \gets m - M_{\floor{v/x}}$
            }
            \Else{
                $m \gets m - M^\prime_{\floor{n/\floor{v/x}}}$
            }
        }
        $M^\prime_y \gets M^\prime_y + m$
        
        $Y \gets Y + y \cdot M^\prime_y$ \label{4-35}
    }
    \end{multicols}
    
    \KwRet $X + Y - Z$
}
\end{algorithm}

The next step is to move line \ref{4-27} into phase 1.  The work this line does is essentially

\begin{algorithm}[H] \label{Algo4extract}
\DontPrintSemicolon
\caption{An extract from Algorithm \ref{Algo4}}
\Begin{
    \For{$y=b$ \KwTo $1$}{
        $M^\prime_y \gets M^\prime_y + 1 - \floor{n/y} + \isqrt{n/y} \cdot M_{\isqrt{n/y}}$
    }
}
\end{algorithm}

This is equivalent to

\begin{algorithm}[H] \label{Algo4extract_redone}
\DontPrintSemicolon
\caption{Algorithm \ref{Algo4extract}, redone}
\Begin{
    \For{$x=1$ \KwTo $a$}{
        \If{$\exists y \;\ni\; 1 \leq y \leq b \;\;\&\;\; x=\isqrt{n/y}$}{
            \For{all such $y$}{
                $M^\prime_y \gets M^\prime_y + 1 - \floor{n/y} + x \cdot M_x$
            }
        }
    }
}
\end{algorithm}

\begin{algorithm}[H] \label{Algo4extract_redone_again}
\DontPrintSemicolon
\caption{Algorithm \ref{Algo4extract}, redone again}
\Begin{
    
    $d \gets b$
    
    $\gamma \gets \isqrt{n/d}$
    
    \For{$x=1$ \KwTo $a$}{
        
        \While{$x = \gamma$}{
            $M^\prime_d \gets M^\prime_d + 1 - \floor{n/d} + x \cdot M_x$
            
            $d \gets d - 1$
            
            $\gamma \gets \isqrt{n/d}$
        }
    }
}
\end{algorithm}

Applying this edit to Algorithm \ref{Algo4} yields Algorithm \ref{Algo8}.  Lines \ref{4-2} and \ref{4-5} have had actions added to them, lines \ref{8-18}--\ref{8-21} have been inserted, and line \ref{4-27} has been replaced with line \ref{8-31}.

\begin{algorithm}[H] \label{Algo8}
\DontPrintSemicolon
\caption{Compute $\Phi(n)$ in $\softTheta(n^{2/3})$ time and $\softTheta(n^{1/2})$ space.  See file \texttt{totientsumE.py} for an implementation.}
\KwData{$n \geq 1$}
\KwResult{$\Phi(n)$}
\Begin{
    $a \gets \floor{\softTheta(n^{2/3})}$; $b \gets \floor{n/a}$; $X \gets 0$; $Y \gets 0$; $Z \gets 0$; $m \gets 0$; $s \gets \isqrt{n}$; $d \gets b$
    
    \lIf{$\isqrt{n} = \floor{n/\isqrt{n}}$}{$s \gets s-1$}
    
    $\chi \gets \floor{n/s}$; $\gamma \gets \isqrt{n/d}$
    
    Prepare a segmented Sieve of Eratosthenes to compute $\mu(x)$ for $1 \leq x \leq a$.
    
    Let $M$ be an array indexed from $1$ through $\isqrt{n}$, inclusive. \label{8-7}
    
    Let $M^\prime$ be an array indexed from $1$ through $\floor{n/\isqrt{n}}$, inclusive, initialized to all zeros.
    \begin{multicols}{2}
    \For{$x=1$ \KwTo $a$}{
        $v \gets \floor{n/x}$
        
        $m \gets m + \mu(x)$
        
        $X \gets X + \mu(x) \cdot \dfrac{v \cdot (v+1)}{2}$
        
        \uIf{$x \leq \isqrt{n}$}{
            $M_x \gets m$ \label{8-14}
            
            \If{$x > 1$}{
                \For{$y=1$ \KwTo $\min(b,\floor{n/x^2})$}{
                    $M^\prime_y \gets M^\prime_y - \mu(x) \cdot \dfloordiv{n}{yx}$
                }
            }
            \While{$x = \gamma$}{ \label{8-18}
                $M^\prime_d \gets M^\prime_d + 1 - \floor{n/d} + mx$
                
                $d \gets d - 1$
                
                $\gamma \gets \isqrt{n/d}$ \label{8-21}
            }
        }
        \ElseIf{$x = \chi$}{
        
            \lIf{$v \neq b$}{$M^\prime_{v} \gets m$}
            
            $s \gets s-1$
            
            $\chi \gets \floor{n/s}$
        }
        \lIf{$x = a$}{$Z \gets m \cdot \dfrac{b \cdot (b+1)}{2}$}
    }
    
    \emph{lines \ref{8-30}--\ref{8-39} here}
    
    \columnbreak
    
    \For{$y=b$ \KwTo $1$}{ \label{8-30}
        $v \gets \floor{n/y}$
        
        $m \gets 0$ \label{8-31}
        
        \For{$x=2$ \KwTo $\isqrt{v}$}{
            \uIf{$\floor{v/x} \leq \isqrt{n}$}{ \label{8-33}
                $m \gets m - M_{\floor{v/x}}$ \label{8-34}
            }
            \Else{ \label{8-35}
                $m \gets m - M^\prime_{\floor{n/\floor{v/x}}}$
            }
        }
        $M^\prime_y \gets M^\prime_y + m$
        
        $Y \gets Y + y \cdot M^\prime_y$ \label{8-39}
    }
    
    \end{multicols}
    
    \KwRet $X + Y - Z$
}
\end{algorithm}

The final obstacle to removing array $M$ is line \ref{8-34}.  The work this line does is essentially

\begin{algorithm}[H] \label{Algo8_extract}
\DontPrintSemicolon
\caption{An extract from Algorithm \ref{Algo8}}
\Begin{
    \For{$y=b$ \KwTo $1$}{
        $v \gets \floor{n/y}$
        
        $m \gets 0$
        
        \For{$x=2$ \KwTo $\isqrt{v}$}{
            \If{$\floor{v/x} \leq \isqrt{n}$}{
                $m \gets m - M_{\floor{v/x}}$
            }
        }
        $M^\prime_y \gets M^\prime_y + m$
    }
}
\end{algorithm}

This extract iterates over all pairs $(x,y)$ with
\neqn{1 \leq y \leq b \qquadtext{and} 2 \leq x \leq \isqrt{n/y} \qquadtext{and} \floor{n/(xy)} \leq \isqrt{n} \label{bafnehkj}}
and, for each such pair, subtracts $M_{\floor{n/(xy)}}$ from $M^\prime_y$.  Note that the third inequality is equivalent to this action all happening during phase 1.  Let $k = \floor{n/(xy)}$.  Then for each Mertens value $M(k)$ that we compute, we must find all pairs of integers $(x,y)$ subject to the above bounds and
\eqn{k \leq \frac{n}{xy} < k+1,}
or equivalently,
\eqn{\frac{n}{k+1} < xy \leq \frac{n}{k}.}
Addressing a single $k$ at a time is awfully close to factoring $\floor{n/k}$.  To avoid breaking the clock, we will instead gather a block of consecutive Mertens values and address them all at once.  When this algorithm is fully developed, the memory usage will be $\softO(\sqrt[3]{n})$ due to the array $M^\prime$ and storage inside the M\"{o}bius siever; we will therefore gather Mertens batches of size $b$.  The high index of each batch will be $x$, and the low index will be $A \defeq 1 + b \cdot \floor{x/b}$.  The result is that, when processing each batch, we will be looking for all pairs $(t, \ell)$ such that
\eqn{1 \leq t \leq b \qquadtext{and} 2 \leq \ell \leq \sqrt{n/t} \qquadtext{and} A = b \cdot \floordiv{x}{b} + 1 \qquadtext{and} A \leq \floordiv{n}{\ell t} \leq x.}
Since $A$ and $x$ are integers, the rightmost condition is equivalent to
\eqn{A \leq \frac{n}{\ell t} < x + 1}
\eqn{\frac{1}{x+1} < \frac{\ell t}{n} \leq \frac{1}{A}}
\eqn{\frac{n}{t \cdot (x+1)} < \ell \leq \frac{n}{At}.}
Furthermore, since $t \leq b$ and the relevant $x$-values are $\leq \isqrt{n}$, the lesser side of this inequality is at least $\softTheta(n^{1/6})$; therefore, the restriction $2 \leq \ell$ above is superfluous.

Algorithm \ref{Algo8_extract} is therefore equivalent to

\begin{algorithm}[H] \label{Algo8_extract_redone}
\DontPrintSemicolon
\caption{Algorithm \ref{Algo8_extract}, redone}
\Begin{
    \For{$x=1$ \KwTo $a$}{
        \If{$x \leq \sqrt{n}$}{
            $\mathcal{M}_x \gets m$
        
        \If{$b \mid x$, or $x = \isqrt{n}$,}{
            Let $A$ be the least index in $\mathcal{M}$.
            
            \For{$t=1$ \KwTo $b$}{
                $\ell_{min} \gets 1 + \dfloordiv{n}{t \cdot (x+1)}$
                
                $\ell_{max} \gets \min\left( \isqrt{n/t} , \dfloordiv{n}{t \cdot A} \right)$
                
                \For{$\ell=\ell_{min}$ \KwTo $\ell_{max}$}{
                    $M^\prime_t \gets M^\prime_t - \mathcal{M}_{\floor{n/(\ell t)}}$
                }
            }
            
            Forget the contents of $\mathcal{M}$.
        }
        }
    }
}
\end{algorithm}

Applying this edit to Algorithm \ref{Algo8} yields Algorithm \ref{Algo11}.  Lines \ref{8-7} and \ref{8-14} have been deleted, lines \ref{8-33}--\ref{8-35} have been replaced with \ref{11-41}, and lines \ref{11-8}, \ref{11-14}, and \ref{11-22}--\ref{11-29} have been inserted.

\begin{algorithm}[H] \label{Algo11}
\DontPrintSemicolon \footnotesize
\caption{Compute $\Phi(n)$ in $\softTheta(n^{2/3})$ time and $\softTheta(n^{1/2})$ space.  See file \texttt{totientsumG.py} for an implementation.}
\KwData{$n \geq 1$}
\KwResult{$\Phi(n)$}
\Begin{
    $a \gets \floor{\softTheta(n^{2/3})}$; $b \gets \floor{n/a}$; $X \gets 0$; $Y \gets 0$; $Z \gets 0$; $m \gets 0$; $s \gets \isqrt{n}$; $d \gets b$
    
    \lIf{$\isqrt{n} = \floor{n/\isqrt{n}}$}{$s \gets s-1$}
    
    $\chi \gets \floor{n/s}$; $\gamma \gets \isqrt{n/d}$
    
    Prepare a segmented Sieve of Eratosthenes to compute $\mu(x)$ for $1 \leq x \leq a$.
    
    Let $M^\prime$ be an array indexed from $1$ through $\floor{n/\isqrt{n}}$, inclusive, initialized to all zeros.
    
    Let $\mathcal{M}$ be an array of size $b$.  Its indexing will vary as the algorithm executes. \label{11-8}
    
    \begin{multicols}{2}
    \For{$x=1$ \KwTo $a$}{
        $v \gets \floor{n/x}$
        
        $m \gets m + \mu(x)$
        
        $X \gets X + \mu(x) \cdot \dfrac{v \cdot (v+1)}{2}$
        
        \uIf{$x \leq \isqrt{n}$}{
            $\mathcal{M}_x \gets m$ \label{11-14}
            
            \If{$x > 1$}{
                \For{$y=1$ \KwTo $\min(b,\floor{n/x^2})$}{
                    $M^\prime_y \gets M^\prime_y - \mu(x) \cdot \dfloordiv{n}{yx}$
                }
            }
            \While{$x = \gamma$}{
                $M^\prime_d \gets M^\prime_d + 1 - \floor{n/d} + mx$
                
                $d \gets d - 1$
                
                $\gamma \gets \isqrt{n/d}$
            }
            \If{$b \mid x$, or $x = \isqrt{n}$,}{ \label{11-22}
                Let $A$ be the least index in $\mathcal{M}$.
                
                \For{$t=1$ \KwTo $b$}{
                    $\ell_{min} \gets 1 + \dfloordiv{n}{t \cdot (x+1)}$
                    
                    $\ell_{max} \gets \min\left( \isqrt{n/t} , \dfloordiv{n}{t \cdot A} \right)$
                    
                    \For{$\ell=\ell_{min}$ \KwTo $\ell_{max}$}{
                        $M^\prime_t \gets M^\prime_t - \mathcal{M}_{\floor{n/(\ell t)}}$
                    }
                }
                
                Forget the contents of $\mathcal{M}$. \label{11-29}
            }
        }
        \ElseIf{$x = \chi$}{
        
            \lIf{$v \neq b$}{$M^\prime_{v} \gets m$}
            
            $s \gets s-1$
            
            $\chi \gets \floor{n/s}$
        }
        \lIf{$x = a$}{$Z \gets m \cdot \dfrac{b \cdot (b+1)}{2}$}
    }
    
    \emph{lines \ref{11-38}--\ref{11-45} here}
    
    \columnbreak
    
    \For{$y=b$ \KwTo $1$}{ \label{11-38}
        $v \gets \floor{n/y}$
        
        $m \gets 0$
        
        \For{$t=2$ \KwTo $\isqrt{v}$}{
            \If{$\floor{v/t} > \isqrt{n}$}{ \label{11-41}
                $m \gets m - M^\prime_{\floor{n/\floor{v/t}}}$
            }
        }
        $M^\prime_y \gets M^\prime_y + m$
        
        $Y \gets Y + y \cdot M^\prime_y$ \label{11-45}
    }
    
    \end{multicols}
    
    \KwRet $X + Y - Z$
}
\end{algorithm}

The only thing left to do is remove the need to store $M^\prime_k$ for $k > b$.

\begin{algorithm}[H] \label{Algo11_extract}
\DontPrintSemicolon
\caption{An extract from Algorithm \ref{Algo11}}
\Begin{
    \For{$y=b$ \KwTo $1$}{
        $v \gets \floor{n/y}$
        
        $m \gets 0$
        
        \For{$t=2$ \KwTo $\isqrt{v}$}{
            \If{$\floor{v/t} > \isqrt{n}$}{
                $m \gets m - M^\prime_{\floor{n/\floor{v/t}}}$
            }
        }
        $M^\prime_y \gets M^\prime_y + m$
    }
    
}
\end{algorithm}

Let $x$ be the index of a Mertens value that gets saved during phase 2 (so that $\isqrt{n} < x \leq a$).  It gets stored as $M^\prime_\floor{n/x}$, and the $t$- and $y$-values of this loop that touch it are exactly those that satisfy
\eqn{1 \leq y \leq b \quadtext{and} 2 \leq t \leq \isqrt{n/y} \quadtext{and} \isqrt{n} < \floordiv{n}{ty} \quadtext{and} \floordiv{n}{x} = \floordiv{n}{\floor{n/(ty)}}.}
As with the transition from Algorithm \ref{Algo8} to Algorithm \ref{Algo11}, we need to assemble a batch of Mertens values and process them all at once to avoid breaking the clock.  To accommodate this, for each $x$-value such that $M(x)$ gets assembled into the batch, let $w=\floor{n/x}$.  Let $A$ be the greatest $w$-value in the batch and let $B$ be the least.  Then the restrictions on $t$ become
\neqn{2 \leq t \leq \isqrt{n/y} \quadtext{and} \isqrt{n} < \floordiv{n}{ty} \quadtext{and} B \leq \floordiv{n}{\floor{n/(ty)}} \leq A. \label{cbafnlj}}
The result is Algorithm \ref{Algo13}.

\begin{algorithm}[H] \label{Algo13}
\DontPrintSemicolon \scriptsize
\caption{Compute $\Phi(n)$ in $\softTheta(n^{2/3})$ time and $\softTheta(n^{1/3})$ space.  See file \texttt{totientsumI.py} for an implementation.}
\KwData{$n \geq 1$}
\KwResult{$\Phi(n)$}
\Begin{
    $a \gets \floor{\softTheta(n^{2/3})}$; $b \gets \floor{n/a}$; $X \gets 0$; $Y \gets 0$; $Z \gets 0$; $m \gets 0$; $s \gets \isqrt{n}$; $d \gets b$
    
    \lIf{$\isqrt{n} = \floor{n/\isqrt{n}}$}{$s \gets s-1$}
    
    $\chi \gets \floor{n/s}$; $\gamma \gets \isqrt{n/d}$
    
    Prepare a segmented Sieve of Eratosthenes to compute $\mu(x)$ for $1 \leq x \leq a$.
    
    Let $M^\prime$ be an array indexed from $1$ through $b$, inclusive, initialized to all zeros.
    
    Let $\mathcal{M}$ be an array of size $b$.  Its indexing will vary as the algorithm executes.
    
    \vspace{1em}
    
    \begin{multicols}{2}
    \For{$x=1$ \KwTo $a$}{
        $v \gets \floor{n/x}$
        
        $m \gets m + \mu(x)$
        
        $X \gets X + \mu(x) \cdot \dfrac{v \cdot (v+1)}{2}$
        
        \uIf{$x \leq \isqrt{n}$}{
            $\mathcal{M}_x \gets m$
            
            \If{$x > 1$}{ \label{13-14}
                \For{$y=1$ \KwTo $\min(b,\floor{n/x^2})$}{
                    $M^\prime_y \gets M^\prime_y - \mu(x) \cdot \dfloordiv{n}{yx}$ \label{13-16}
                }
            }
            \While{$x = \gamma$}{
                $M^\prime_d \gets M^\prime_d + 1 - \floor{n/d} + mx$ \label{13-18}
                
                $d \gets d - 1$
                
                $\gamma \gets \isqrt{n/d}$ \label{13-20}
            }
            \If{$b \mid x$, or $x = \isqrt{n}$,}{ \label{13-21}
                Let $A$ be the least index in $\mathcal{M}$.
                
                \For{$t=1$ \KwTo $b$}{
                    $\ell_{min} \gets 1 + \dfloordiv{n}{t \cdot (x+1)}$
                    
                    $\ell_{max} \gets \min\left( \isqrt{n/t} , \dfloordiv{n}{t \cdot A} \right)$
                    
                    \For{$\ell=\ell_{min}$ \KwTo $\ell_{max}$}{
                        $M^\prime_t \gets M^\prime_t - \mathcal{M}_{\floor{n/(\ell t)}}$ \label{13-27}
                    }
                }
                
                Forget the contents of $\mathcal{M}$.
            }
        }
        \ElseIf{$x = \chi$}{
        
            \If{$v \neq b$}{
                
                \lIf{$\mathcal{M}$ is empty}{$A \gets v$}
                
                $\mathcal{M}_v \gets m$; $B \gets v$
            }
            
            $s \gets s-1$
            
            $\chi \gets \floor{n/s}$
        }
        \lIf{$x = a$}{$Z \gets m \cdot \dfrac{b \cdot (b+1)}{2}$}
        \If{$x = a$ or ($x > \isqrt{n}$ and $\mathcal{M}$ is full)}{
            \For{$y=1$ to $b$}{
                \ForAll{$t$ satisfying (\ref{cbafnlj})}{
                    $M^\prime_y \gets M^\prime_y - \mathcal{M}_{ty}$ \label{13-39}
                }
            }
            Forget the contents of $\mathcal{M}$.
        }
    }
    
    \emph{lines \ref{13-42}--\ref{13-49} here}
    
    \columnbreak
    
    \For{$y=b$ \KwTo $1$}{ \label{13-42}
        $v \gets \floor{n/y}$
        
        $m \gets 0$
        
        \For{$t=2$ \KwTo $\isqrt{v}$}{ \label{13-45}
            \If{$\floor{v/t} > \isqrt{n}$ and $\floor{n/\floor{v/t}} \leq b$}{ \label{13-46}
                $m \gets m - M^\prime_{\floor{n/\floor{v/t}}}$ \label{13-47}
            }
        }
        $M^\prime_y \gets M^\prime_y + m$
        
        $Y \gets Y + y \cdot M^\prime_y$ \label{13-49}
    }
    
    \end{multicols}
    
    \KwRet $X + Y - Z$
}
\end{algorithm}

We have now hit our target time- and space-complexities, but some further optimization can be done.  In particular, the iteration over $t$ in phase 3---lines \ref{13-45}--\ref{13-47}---can be made more efficient.  The conditions on $x$ are
\eqn{2 \leq x \leq \isqrt{n/y} \qquadtext{and} \floordiv{\floor{n/y}}{x} > \isqrt{n} \qquadtext{and} \floordiv{n}{\floor{\floor{n/y}/x}} \leq b,}
which is equivalent to
\eqn{2 \leq x \leq \isqrt{n/y} \qquadtext{and} \floordiv{\floor{n/y}}{x} \geq \isqrt{n}+1 \qquadtext{and} \floordiv{n}{\floor{\floor{n/y}/x}} \leq b}
\eqn{2 \leq x \leq \isqrt{n/y} \qquadtext{and} \frac{\floor{n/y}}{x} \geq \isqrt{n}+1 \qquadtext{and} \frac{n}{\floor{\floor{n/y}/x}} < b+1}
\eqn{2 \leq x \leq \isqrt{n/y} \qquadtext{and} x \leq \frac{\floor{n/y}}{\isqrt{n}+1} \qquadtext{and} \frac{n}{b+1} < \floordiv{\floor{n/y}}{x}}
Since $x$ is an integer, we can apply the floor function to the upper side of the middle inequality.
\eqn{2 \leq x \leq \isqrt{n/y} \qquadtext{and} x \leq \floor{\frac{\floor{n/y}}{\isqrt{n}+1}} \qquadtext{and} \frac{n}{b+1} < \floordiv{\floor{n/y}}{x}}
By Lemma \ref{lemma1}, we can drop the $x \leq \isqrt{n/y}$ condition, yielding
\eqn{2 \leq x \qquadtext{and} x \leq \floor{\frac{\floor{n/y}}{\isqrt{n}+1}} \qquadtext{and} \frac{n}{b+1} < \floordiv{\floor{n/y}}{x}.}
The rightmost inequality is stricter than the middle inequality, so we are left with
\eqn{2 \leq x \qquadtext{and} \frac{n}{b+1} < \floordiv{\floor{n/y}}{x}.}

A further improvement can be had by inserting ``and $\mu(x) \neq 0$" into line \ref{13-14}.

Making the corresponding edits yields Algorithm \ref{Algo14}.

\begin{algorithm}[H] \label{Algo14}
\DontPrintSemicolon \scriptsize
\caption{Compute $\Phi(n)$ in $\softTheta(n^{2/3})$ time and $\softTheta(n^{1/3})$ space.  See file \texttt{totientsumJ.py} for an implementation.}
\KwData{$n \geq 1$}
\KwResult{$\Phi(n)$}
\Begin{
    $a \gets \floor{\softTheta(n^{2/3})}$; $b \gets \floor{n/a}$; $X \gets 0$; $Y \gets 0$; $Z \gets 0$; $m \gets 0$; $s \gets \isqrt{n}$; $d \gets b$
    
    \lIf{$\isqrt{n} = \floor{n/\isqrt{n}}$}{$s \gets s-1$}
    
    $\chi \gets \floor{n/s}$; $\gamma \gets \isqrt{n/d}$
    
    Prepare a segmented Sieve of Eratosthenes to compute $\mu(x)$ for $1 \leq x \leq a$.
    
    Let $M^\prime$ be an array indexed from $1$ through $b$, inclusive, initialized to all zeros.
    
    Let $\mathcal{M}$ be an array of size $b$.  Its indexing will vary as the algorithm executes.
    
    \begin{multicols}{2}
    \For{$x=1$ \KwTo $a$}{
        $v \gets \floor{n/x}$
        
        $m \gets m + \mu(x)$
        
        $X \gets X + \mu(x) \cdot \dfrac{v \cdot (v+1)}{2}$
        
        \uIf{$x \leq \isqrt{n}$}{
            $\mathcal{M}_x \gets m$
            
            \If{$x > 1$ and $\mu(x) \neq 0$}{
                \For{$y=1$ \KwTo $\min(b,\floor{v/x})$}{
                    $M^\prime_y \gets M^\prime_y - \mu(x) \cdot \dfloordiv{v}{y}$ \label{14-16}
                }
            }
            \While{$x = \gamma$}{
                $M^\prime_d \gets M^\prime_d + 1 - \floor{n/d} + mx$ \label{14-18}
                
                $d \gets d - 1$
                
                $\gamma \gets \isqrt{n/d}$ \label{14-20}
            }
            \If{$b \mid x$, or $x = \isqrt{n}$,}{ \label{14-21}
                Let $A$ be the least index in $\mathcal{M}$.
                
                \For{$t=1$ \KwTo $b$}{
                    $\ell_{min} \gets 1 + \dfloordiv{n}{t \cdot (x+1)}$
                    
                    $\ell_{max} \gets \min\left( \isqrt{n/t} , \dfloordiv{n}{t \cdot A} \right)$
                    
                    \For{$\ell=\ell_{min}$ \KwTo $\ell_{max}$}{
                        $M^\prime_t \gets M^\prime_t - \mathcal{M}_{\floor{n/(\ell t)}}$ \label{14-27}
                    }
                }
                
                Forget the contents of $\mathcal{M}$.
            }
        }
        \ElseIf{$x = \chi$}{
        
            \If{$v \neq b$}{
                
                \lIf{$\mathcal{M}$ is empty}{$A \gets v$}
                
                $\mathcal{M}_v \gets m$; $B \gets v$
            }
            
            $s \gets s-1$
            
            $\chi \gets \floor{n/s}$
        }
        \lIf{$x = a$}{$Z \gets m \cdot \dfrac{b \cdot (b+1)}{2}$}
        \If{$x = a$ or ($x > \isqrt{n}$ and $\mathcal{M}$ is full)}{
            \For{$y=1$ to $b$}{
                \ForAll{$t$ satisfying (\ref{cbafnlj})}{
                    $M^\prime_y \gets M^\prime_y - \mathcal{M}_{ty}$ \label{14-39}
                }
            }
            Forget the contents of $\mathcal{M}$.
        }
    }
    
    \emph{lines \ref{14-42}--\ref{14-50} here}
    
    \columnbreak
    
    \For{$y=b$ \KwTo $1$}{ \label{14-42}
        $v \gets \floor{n/y}$
        
        $m \gets 0$
        
        \For{$x \in \set{2,3,4,...}$}{ \label{14-45}
            \If{$\disp \frac{n}{b+1} \geq \floordiv{v}{x}$}{ \label{14-46}
                \KwBreak
            }
            
            $m \gets m - M^\prime_{\floor{n/\floor{v/x}}}$ \label{14-48}
        }
        $M^\prime_y \gets M^\prime_y + m$
        
        $Y \gets Y + y \cdot M^\prime_y$ \label{14-50}
    }
    
    \end{multicols}
    
    \KwRet $X + Y - Z$
}
\end{algorithm}



\subsection{Analysis of Algorithm \ref{Algo14}}

The lines that (have the potential to) dominate the execution time are the M\"{o}bius siever, \ref{14-16}, \ref{14-18}--\ref{14-20}, \ref{14-27}, \ref{14-39}, and \ref{14-46}--\ref{14-48}.

\begin{lemma} \label{14-M-time}
The M\"{o}bius sieving consumes $\Theta(a \ln(\ln(a)))$ time.
\end{lemma}
\begin{proof}
The M\"{o}bius function is sieved up to $a$; the rest is a standard result.
\end{proof}

\begin{lemma} \label{14-16-time}
Line \ref{14-16} consumes $\Theta(n/\sqrt{a})$ time.
\end{lemma}
\begin{proof}
Line \ref{14-16} is hit
\eqn{\sum_{\substack{2 \leq x \leq \isqrt{n} \\ \mu(x) \neq 0}} \min\left(\floordiv{n}{a}, \floordiv{n}{x^2}\right)}
times.  Since the squarefree integers have a natural density of $6/\pi^2$, this is
\eqn{= \Theta\left( \sum_{x=2}^{\isqrt{n}} \min\left(\floordiv{n}{a}, \floordiv{n}{x^2}\right) \right)}
\eqn{= \Theta\left( \sum_{x=2}^{\isqrt{a}} \floordiv{n}{a} + \sum_{x=\isqrt{a}+1}^{\isqrt{n}} \floordiv{n}{x^2} \right)}
\eqn{= \Theta\left( \floordiv{n}{a} \cdot \left(\isqrt{a} - 1\right) + \integral{\;\frac{n}{x^2}}{x}{\isqrt{a}+1}{\isqrt{n}}\right)}
\eqn{= \Theta\left(\floordiv{n}{a} \cdot \left(\isqrt{a} - 1\right) - n \cdot \eval{x^{-1}}{x=\isqrt{a}+1}{\isqrt{n}} \right)}
\eqn{= \Theta\left( \floordiv{n}{a} \cdot \left(\isqrt{a} - 1\right) - n \cdot \left(\frac{1}{\isqrt{n}} - \frac{1}{\isqrt{a}+1}\right)\right)}
\eqn{= \Theta\left(\floordiv{n}{a} \cdot \left(\isqrt{a} - 1\right) + \frac{n}{\isqrt{a}+1} - \frac{n}{\isqrt{n}}\right)}
\eqn{= \Theta\left(\frac{n}{\isqrt{a}} + \frac{n}{\isqrt{a}+1} - \frac{n}{\isqrt{n}}\right)}
\eqn{= \Theta\left(\frac{n}{\sqrt{a}} - \sqrt{n}\right)}
\eqn{= \Theta\left(\frac{n}{\sqrt{a}}\right)}
\end{proof}

\begin{lemma} \label{Algo14mintime}
The runtime of Algorithm \ref{Algo14} is at least $\Theta(n^{2/3} (\ln(\ln(n)))^{1/3})$.
\end{lemma}
\begin{proof}
By Lemmas \ref{14-M-time} and \ref{14-16-time}, the combined runtime of of the M\"{o}bius sieving and line \ref{14-16} is
\eqn{\Theta(a \ln(\ln(a))) + \Theta\left(\frac{n}{\sqrt{a}}\right),}
which is minimized by choosing
\eqn{a = \Theta\left(\left(\frac{n}{\ln(\ln(n))}\right)^{2/3}\right);}
the runtime of those parts is then $\Theta\left(n^{2/3} (\ln(\ln(n)))^{1/3}\right)$ each.
\end{proof}

Lines \ref{14-18}--\ref{14-20} can be neglected: they get hit at most $b$ times, and the time per hit is dominated by the computation of a single square root.

\begin{lemma} \label{14-27-time}
With $a = \Theta((n/\ln(\ln(n)))^{2/3})$, line \ref{14-27} consumes $\Theta\left(n^{2/3} (\ln(\ln(n)))^{1/3}\right)$ time.
\end{lemma}
\begin{proof}
Line \ref{14-27} gets hit at least
\eqn{\sum_{\substack{1 \leq x \leq \sqrt{n} \\ b \mid x}}\sum_{t=1}^b \left( 1 + \max\left( \ell_{max}(x,t) - \ell_{min}(x,t), 0 \right) \right)}
times.  If $b \mid \isqrt{n}$, then this is exactly the number of hits; otherwise, there will be a final phase-1 batch that is not included in the above sum.

The first batch ($x=b$) and that possible last batch $(x=\isqrt{n})$ need special handling.  For the first batch, when $x=b$, the relevant section of the algorithm amounts to

\begin{algorithm}[H]
\DontPrintSemicolon
\caption{An extract from Algorithm \ref{Algo14}, with $x=b$}
\Begin{
    \For{$t=1$ \KwTo $b$}{
        $\ell_{min} \gets 1 + \dfloordiv{n}{t \cdot (b+1)}$
        
        $\ell_{max} \gets \min\left( \isqrt{n/t} , \dfloordiv{n}{t} \right)$
        
        \For{$\ell=\ell_{min}$ \KwTo $\ell_{max}$}{
            $M^\prime_t \gets M^\prime_t - \mathcal{M}_{\floor{n/(\ell t)}}$
        }
    }
}
\end{algorithm}

The time consumed by this extract is
\eqn{\sum_{t=1}^b \left( 1 + \max\left( \ell_{max} - \ell_{min} , 0 \right) \right)}
\eqn{= \sum_{t=1}^b \left( 1 + \max\left( \min\left( \isqrt{n/t} , \floordiv{n}{t} \right) - 1 - \floordiv{n}{t \cdot (b+1)} , 0 \right) \right)}
\eqn{= b + \sum_{t=1}^b \max\left( \min\left( \isqrt{n/t} , \floordiv{n}{t} \right) - 1 - \floordiv{n}{t \cdot (b+1)} , 0 \right)}
We have $0 < t < n$, so $\sqrt{n/t} \leq n/t$.
\eqn{= b + \sum_{t=1}^b \max\left( \isqrt{n/t} - 1 - \floordiv{n}{t \cdot (b+1)} , 0 \right)}
For $b \leq n^{1/3}$, Lemma \ref{ouawrt4coi} establishes that the first argument of the $\max$ function is negative, so the time consumed by the extract is $O(b)$.  Unfortunately, it will turn out that $b > n^{1/3}$.  In this case, the $\max$ function's first argument prevails for
\eqn{t > t_0 \defeq n \cdot \left(\frac{1}{2} - \frac{1}{b+1} - \sqrt{\frac{1}{4} - \frac{1}{b+1}}\right),}
so the time consumed by the extract is
\eqn{b + \sum_{t=t_0}^b \left( \isqrt{n/t} - 1 - \floordiv{n}{t \cdot (b+1)} \right)}
\eqn{\approx b + \int_{t_0}^b \left( \sqrt{\frac{n}{t}} - 1 - \frac{n}{t \cdot (b+1)} \right) dt}
\eqn{= b + \eval{\left( 2\sqrt{nt} - t - \frac{n}{b+1} \ln(\abs{t}) \right)}{t=t_0}{b}}
\eqn{= b + 2\sqrt{bn} - 2\sqrt{nt_0} - b + t_0 - \frac{n}{b+1}\ln\left(\frac{b}{t_0}\right)}
\eqn{= 2\sqrt{bn} - 2\sqrt{nt_0} + t_0 - \frac{n}{b+1}\ln\left(\frac{b}{t_0}\right)}
From Lemma \ref{sqrtapprox2}, we have that $t_0 = n \cdot (b^{-2} + O(b^{-3}))$.  Therefore,
\eqn{= 2\sqrt{bn} - 2\sqrt{n^2 \cdot (b^{-2} + O(b^{-3}))} + n \cdot (b^{-2} + O(b^{-3})) - \frac{n}{b+1}\ln\left(\frac{b}{n^2 \cdot (b^{-2} + O(b^{-3}))}\right)}
\eqn{\approx 2\sqrt{bn} - 2\sqrt{n^2 \cdot b^{-2}} + n \cdot b^{-2} - \frac{n}{b}\ln\left(\frac{b^3}{n^2}\right)}
\eqn{= 2\frac{n}{\sqrt{a}} - 2a - a\ln\left(\frac{n^2}{a^3}\right).}

In the last batch, when $x = \isqrt{n} \not\equiv 0 \pmod{b}$, the relevant section of the algorithm amounts to

\begin{algorithm}[H]
\DontPrintSemicolon
\caption{An extract from Algorithm \ref{Algo14}, with $x = \isqrt{n} \not\equiv 0 \pmod{b}$}
\Begin{
    $x \gets \isqrt{n}$
    
    $A \gets 1 + b \cdot \floor{x/b}$
    
    \For{$t=1$ \KwTo $b$}{
        $\ell_{min} \gets 1 + \dfloordiv{n}{t \cdot (x+1)}$
        
        $\ell_{max} \gets \min\left( \isqrt{n/t} , \dfloordiv{n}{t \cdot A} \right)$
        
        \For{$\ell=\ell_{min}$ \KwTo $\ell_{max}$}{
            $M^\prime_t \gets M^\prime_t - \mathcal{M}_{\floor{n/(\ell t)}}$
        }
    }
}
\end{algorithm}

The time consumed by this extract is
\eqn{\sum_{t=1}^b \left( 1 + \max\left( \ell_{max} - \ell_{min} , 0 \right) \right)}
\eqn{= b + \sum_{t=1}^b \max\left( \min\left( \isqrt{n/t} , \dfloordiv{n}{t \cdot A} \right) - 1 - \dfloordiv{n}{t \cdot (x+1)} , 0 \right)}
\neqn{= b + \sum_{t=1}^b \max\left( \min\left( \isqrt{n/t} , \dfloordiv{n}{t \cdot \left(1 + b \cdot \dfloordiv{\isqrt{n}}{b}\right)} \right) - 1 - \dfloordiv{n}{t \cdot (\isqrt{n}+1)} , 0 \right) \label{hydrogen}}
We split off the $t=1$ term for special treatment: this is
\eqn{\max\left( \min\left( \isqrt{n/1} , \dfloordiv{n}{1 \cdot \left(1 + b \cdot \dfloordiv{\isqrt{n}}{b}\right)} \right) - 1 - \dfloordiv{n}{1 \cdot (\isqrt{n}+1)} , 0 \right)}
\eqn{= \max\left( \min\left( \isqrt{n} , \dfloordiv{n}{1 + b \cdot \dfloordiv{\isqrt{n}}{b}} \right) - 1 - \dfloordiv{n}{\isqrt{n}+1} , 0 \right)}
By hypothesis, we are working with $\isqrt{n} \not\equiv 0 \pmod{b}$, so $1 + b \cdot \floor{\isqrt{n}/b} \leq \isqrt{n}$.  Therefore, in the $\min$ function, the first argument prevails.
\eqn{= \max\left( \isqrt{n} - 1 - \dfloordiv{n}{\isqrt{n}+1} , 0 \right)}
\eqn{= 0}
Therefore (\ref{hydrogen}) becomes
\eqn{= b + \sum_{t=2}^b \max\left( \min\left( \isqrt{n/t} , \dfloordiv{n}{t \cdot \left(1 + b \cdot \dfloordiv{\isqrt{n}}{b}\right)} \right) - 1 - \dfloordiv{n}{t \cdot (\isqrt{n}+1)} , 0 \right)}
By Lemma \ref{okiagfvwr}, in the $\min$ function, the second argument prevails.
\eqn{= b + \sum_{t=2}^b \max\left( \floordiv{n}{t \cdot \left(1 + b \cdot \dfloordiv{\isqrt{n}}{b}\right)} - 1 - \floordiv{n}{t \cdot (\isqrt{n}+1)} , 0 \right)}
\eqn{= \Theta\left( b + \int_2^b \max\left( \frac{n}{t \cdot \left(1 + b \cdot \dfloordiv{\isqrt{n}}{b}\right)} - 1 - \frac{n}{t \cdot (\isqrt{n}+1)} , 0 \right) dt \right)}
Let
\eqn{U \defeq n \cdot \frac{\isqrt{n} - b \cdot \dfloordiv{\isqrt{n}}{b}}{\left(1 + b \cdot \dfloordiv{\isqrt{n}}{b}\right)\left(\isqrt{n}+1\right)}.}
Then
\eqn{= \Theta\left( b + \int_2^b \max\left( \frac{U}{t} - 1 , 0 \right) dt \right)}
\eqn{< \Theta\left( b + \int_2^b \max\left( \frac{U}{t} , 0 \right) dt \right)}
\eqn{< \Theta\left( b + U \ln(U) \right)}
By Lemma \ref{lmmfanb}, this can be weakened to
\eqn{= O(b + \sqrt{n} \ln(n)).}

Recall that we are in the process of estimating the time devoted to line \ref{14-27}.  We have separated out the first and last batches for special treatment, and found them to consume
\eqn{2\frac{n}{\sqrt{a}} - 2a + \frac{a^2}{n} - a\ln\left(\frac{n^2}{a^3}\right) + O(b + \sqrt{n} \ln(n))}
time.  With $a = \Theta((n/\ln(\ln(n)))^{2/3})$, this becomes
\eqn{2\frac{n}{\sqrt{\Theta\left(\left(\dfrac{n}{\ln(\ln(n))}\right)^{2/3}\right)}} - 2a + \frac{\softTheta(n^{2/3})^2}{n} - a\ln\left(\frac{n^2}{\Theta\left(\left(\dfrac{n}{\ln(\ln(n))}\right)^{2/3}\right)^3}\right) + O(b + \sqrt{n} \ln(n))}
\eqn{= 2\frac{n}{\Theta\left(\left(\dfrac{n}{\ln(\ln(n))}\right)^{1/3}\right)} - 2a + \softTheta(n^{1/3}) - a\ln\left(\frac{n^2}{\Theta\left(\left(\dfrac{n}{\ln(\ln(n))}\right)^2\right)}\right) + O(b + \sqrt{n} \ln(n))}
\eqn{= \Theta\left(n^{2/3} (\ln(\ln(n)))^{1/3}\right) - 2a + \softTheta(n^{1/3}) - a \Theta(\ln(\ln(\ln(n)))) + O(b + \sqrt{n} \ln(n))}
\eqn{= \Theta(a \ln(\ln(n))) - 2a + \softTheta(n^{1/3}) - \Theta(a \ln(\ln(\ln(n)))) + O(b + \sqrt{n} \ln(n))}
\neqn{= \Theta(a \ln(\ln(n))). \label{14-27-time-1}}

We now turn our attention to those batches in which $b \mid x$ and $x \neq b$.  For those batches, we have $A = 1 + x - b$, so the time devoted to those batches is

\eqn{\sum_{\substack{2b \leq x \leq \sqrt{n} \\ b \mid x}}\sum_{t=1}^b \left( 1 + \max\left( \min \left( \isqrt{n/t} , \floordiv{n}{t \cdot (1 + x - b)} \right)
-
\left( 1 + \floordiv{n}{t \cdot (x+1)} \right)
, 0 \right) \right).}

By Lemma \ref{BigIntegral}, this is
\eqn{\approx a \cdot \left( \ln(4) + \ln(2)^2 - \ln(2) \ln(n) + \ln(8)\ln(n/a) \right).}
With $a = \Theta\left((n/\ln(\ln(n)))^{2/3}\right)$, this works out to
\eqn{= a \cdot \left( \ln(4) + \ln(2)^2 - \ln(2) \ln(n) + \ln(8)\ln\left( \frac{n}{\Theta\left(\left(\dfrac{n}{\ln(\ln(n))}\right)^{2/3}\right)} \right)\right)}
\eqn{= a \cdot \left( \ln(4) + \ln(2)^2 - \ln(2) \ln(n) + \ln(8)\ln\left( \Theta\left( n^{1/3} (\ln(\ln(n)))^{2/3} \right)\right)\right)}
\eqn{= a \cdot \left( \ln(4) + \ln(2)^2 - \ln(2) \ln(n) + \frac{\ln(8)}{3}\ln\left( \Theta\left( n (\ln(\ln(n)))^2 \right)\right)\right)}
\eqn{= a \cdot \left( \ln(4) + \ln(2)^2 - \ln(2) \ln(n) + \ln(2)\ln\left( \Theta\left( n (\ln(\ln(n)))^2 \right)\right)\right)}
\eqn{= a \cdot \left( \ln(4) + \ln(2)^2 - \ln(2) \ln(n) + \ln(2)\ln\left( n \cdot \Theta\left( (\ln(\ln(n)))^2 \right)\right)\right)}
\eqn{= a \cdot \left( \ln(4) + \ln(2)^2 - \ln(2) \ln(n) + \ln(2)\ln(n) + \ln(2)\ln\left( \Theta\left( (\ln(\ln(n)))^2 \right)\right) \right)}
\eqn{= a \cdot \left( \ln(4) + \ln(2)^2 + \ln(2)\ln\left( \Theta\left( (\ln(\ln(n)))^2 \right)\right) \right)}
\eqn{= \ln(2) \cdot a \cdot \left( 2 + \ln(2) + \ln\left( \Theta\left( (\ln(\ln(n)))^2 \right)\right) \right)}
\eqn{= \ln(2) \cdot a \cdot \left( \ln(2e^2) + \ln\left( \Theta\left( (\ln(\ln(n)))^2 \right)\right) \right)}
\eqn{= \ln(2) \cdot a \cdot \left( \ln(2e^2) + 2\ln\left( \Theta\left( \ln(\ln(n)) \right)\right) \right)}
\eqn{= \ln(2) \cdot a \cdot \left( \ln(2e^2) + 2\ln\left( \Theta(1) \cdot \ln(\ln(n)) \right) \right)}
\eqn{= \ln(2) \cdot a \cdot \left( \ln(2e^2) + 2 \ln(\Theta(1)) + 2\ln\left( \ln(\ln(n)) \right) \right)}
\eqn{= \ln(2) \cdot a \cdot \left( \Theta(1) + 2\ln\left( \ln(\ln(n)) \right) \right)}
\neqn{= \Theta\left(a \cdot \ln(\ln(\ln(n)))\right) \label{14-27-time-2}}
From (\ref{14-27-time-1}) and (\ref{14-27-time-2}), we see that, with $a = \Theta((n/\ln(\ln(n)))^{2/3})$, the total time consumed in handling line \ref{14-27} is
\eqn{\Theta(a \cdot \ln(\ln(n))) + \Theta(a \cdot \ln(\ln(\ln(n))))}
\eqn{= \Theta(a \cdot \ln(\ln(n)))}
\eqn{= \Theta \left( n^{2/3} (\ln(\ln(n)))^{1/3} \right).}
\end{proof}
\begin{lemma} \label{14-39-time}
Line \ref{14-39} consumes $\Theta\left(\sqrt{n} \ln(b)\right)$ time.
\end{lemma}
\begin{proof}
On each iteration through phase 2 in which line \ref{14-39} gets hit, it gets hit
\eqn{\sum_{y=1}^b f(t,y)}
times, where $f(t,y)$ is the number of integers $t$ that satisfy (\ref{cbafnlj}).

We need to determine which values of $A$ and $B$ happen.  The $k$\textsuperscript{th} value of $x$ such that $M(x)$ gets saved during phase 2 is approximately
\eqn{x_k \defeq \frac{n}{\sqrt{n} - k}.}
We accumulate batches of size $b$, so a batch will be processed when $k \mid b$; the first batch has $k=b$ and the last batch will have $k \approx \sqrt{n} - \sqrt[3]{n}$.  The $j$\textsuperscript{th} batch will then have
\eqn{A \approx \sqrt{n} - b \cdot (j-1) \qquadtext{and} B \approx \sqrt{n} - bj.}
When processing the $j$\textsuperscript{th} batch, the bounds on $t$ are therefore approximately
\eqn{2 \leq t \leq \sqrt{n/y} \quadtext{and} \sqrt{n} < \frac{n}{ty} \quadtext{and} \sqrt{n}-bj \leq ty \leq \sqrt{n}-b\cdot(j-1)}
\eqn{2 \leq t \leq \sqrt{n/y} \quadtext{and} t < \frac{\sqrt{n}}{y} \quadtext{and} \frac{\sqrt{n}-bj}{y} \leq t \leq \frac{\sqrt{n}-bj}{y}+\frac{b}{y}}
The second condition and the upper side of the first condition are both weaker than the upper side of the third condition.
\eqn{2 \leq t \quadtext{and} \frac{\sqrt{n}-bj}{y} \leq t \leq \frac{\sqrt{n}-bj}{y}+\frac{b}{y}}
Therefore
\eqn{f(t,y) \approx \frac{\sqrt{n}-bj}{y}+\frac{b}{y} - \max\left(2, \frac{\sqrt{n}-bj}{y}\right),}
so on an iteration through phase 2 in which line \ref{14-39} gets hit, it gets hit
\eqn{\Theta\left(\sum_{y=1}^b \left( \frac{\sqrt{n}-bj}{y}+\frac{b}{y} - \max\left(2, \frac{\sqrt{n}-bj}{y}\right)\right)\right) = \Theta\left( \sum_{y=1}^b \frac{b}{y} \right) = \Theta\left( b \ln(b) \right)}
times.
There are $\Theta(\sqrt{n}-\sqrt[3]{n})$ values of $x$ such that $M(x)$ gets saved during phase 2; since each batch has size $b$, there are
\eqn{\Theta\left( \frac{\sqrt{n}-\sqrt[3]{n}}{b} \right)}
batches.  The total time devoted to line \ref{14-39} across all iterations through phase 2 is therefore
\eqn{\Theta\left( b \ln(b) \right) \cdot \Theta\left( \frac{\sqrt{n}-\sqrt[3]{n}}{b} \right)}
\eqn{= \Theta\left( b \ln(b) \right) \cdot \Theta\left( \frac{\sqrt{n}}{b} \right)}
\eqn{= \Theta\left( \sqrt{n} \ln(b) \right).}
\end{proof}

\begin{lemma} \label{14-46-48-time}
Lines \ref{14-46}--\ref{14-48} consume $\Theta(b \ln(b))$ time.
\end{lemma}
\begin{proof}
These lines get hit
\eqn{\Theta\left( \sum_{y=1}^b \frac{n/y}{n/(b+1)} \right) = \Theta\left( \sum_{y=1}^b \frac{b+1}{y} \right) = \Theta\left( b \ln(b) \right)}
times.
\end{proof}

\begin{theorem} \label{Algo14time}
Algorithm \ref{Algo14}'s time complexity is optimized by choosing
\eqn{a = \Theta\left(\left( \frac{n}{\ln(\ln(n))} \right)^{2/3} \right),}
which results in $\Theta\left(n^{1/3} (\ln(\ln(n)))^{2/3}\right)$ space and $\Theta\left( n^{2/3} (\ln(\ln(n)))^{1/3} \right)$ time.
\end{theorem}
\begin{proof}
The space complexity follows by inspection; the time complexity follows immediately from Lemmas \ref{14-M-time}, \ref{14-16-time}, \ref{Algo14mintime}, \ref{14-27-time}, \ref{14-39-time}, and \ref{14-46-48-time}.
\end{proof}

\section{Python code}

%\showcode{totientsumA}

Filename: \texttt{utils.py}
%\lstinputlisting[language=Python, caption={Utiliity functions---a prime-number generator, a $\floor{\isqrt[n]{x}}$-computer, a M\"{o}bius siever, and the Mertens function using the Del\'{e}glise--Rivat algorithm.}]{code/utils.py}

Filename: \texttt{totientsumA.py}
%\lstinputlisting[language=Python, caption={Computation of $\Phi(N)$ by sieving $\phi$ from 1 to $N$.}]{code/totientsumA.py}

Filename: \texttt{totientsumB.py}
%\lstinputlisting[language=Python, caption={Computation of $\Phi(N)$ by na\"{i}vely evaluating (\ref{PhiFormula}).}]{code/totientsumB.py}

Filename: \texttt{totientsumC.py}
%\lstinputlisting[language=Python, caption={Computation of $\Phi(N)$ by Algorithm \ref{Algo1}.}]{code/totientsumC.py}

Filename: \texttt{totientsumD.py}
%\lstinputlisting[language=Python, caption={Computation of $\Phi(N)$ by Algorithm \ref{Algo4}.}]{code/totientsumD.py}

Filename: \texttt{totientsumE.py}
%\lstinputlisting[language=Python, caption={Computation of $\Phi(N)$ by Algorithm \ref{Algo8}.}]{code/totientsumE.py}

Filename: \texttt{totientsumF.py}
%\lstinputlisting[language=Python, caption={Computation of $\Phi(N)$ by an unbatched version of Algorithm \ref{Algo11}.}]{code/totientsumF.py}

Filename: \texttt{totientsumG.py}
%\lstinputlisting[language=Python, caption={Computation of $\Phi(N)$ by Algorithm \ref{Algo11}.}]{code/totientsumG.py}

Filename: \texttt{totientsumI.py}
%\lstinputlisting[language=Python, caption={Computation of $\Phi(N)$ by Algorithm \ref{Algo13}.}]{code/totientsumI.py}

Filename: \texttt{totientsumJ.py}
%\lstinputlisting[language=Python, caption={Computation of $\Phi(N)$ by Algorithm \ref{Algo14}.}]{code/totientsumJ.py}

Filename: \texttt{totientsumK.py}
%\lstinputlisting[language=Python, caption={Computation of $\Phi(N)$ by Algorithm \ref{Algo14}.}]{code/totientsumK.py}

\section{Computational results}

The following data was procured by running \texttt{totientsumK.py} under PyPy3 on a Linux machine with an AMD Ryzen 9 7950X processor and 128 GB of RAM.  The time and space requirements are the wall-clock time and maximum resident set size reported by the \texttt{/usr/bin/time -v} command.

\begin{tabular}{|c|r|r|r|} \hline
$n$ & $\Phi(n)$ & time (s) & memory (kb) \\\hline
$10^{13}$ & 30396355092702898919527444 & 24 & 95048 \\\hline
$10^{14}$ & 3039635509270144893910357854 & 109 & 120508 \\\hline
$10^{15}$ & 303963550927013509478708835152 & 517 & 195252 \\\hline
$10^{16}$ & 30396355092701332166351822199504 & 2466 & 306988\\\hline
$10^{17}$ & 3039635509270133156701800820366346 & 12145 & 717536 \\\hline
$10^{18}$ & 303963550927013314319686824781290348 & 57601 & 1688880 \\\hline
$10^{19}$ &  &  &  \\\hline
\end{tabular}

Based on the algorithm's time-complexity, one would expect the computation of $\Phi(10^{19})$ to take about 267,000 seconds.  The reason for the much longer time is presumably that $n$ has ceased to be a 64-bit number.

\section{Appendix}

\begin{lemma} \label{lemma1}
Let $n$ and $y$ be integers such that $1 \leq y \leq \sqrt{n}$.  Then
\eqn{\floordiv{\floor{n/y}}{\isqrt{n}+1} \leq \isqrt{\frac{n}{y}}.}
\end{lemma}
\begin{proof}
We begin with $\sqrt{n} \leq \isqrt{n} + 1$.  We can then weaken this to
\eqn{\sqrt{\frac{n}{y}} \leq \isqrt{n} + 1}
\eqn{\frac{\sqrt{n/y}}{\isqrt{n}+1} \leq 1}
\eqn{\frac{n/y}{\isqrt{n}+1} \leq \sqrt{\frac{n}{y}}}
\eqn{\frac{\floor{n/y}}{\isqrt{n}+1} \leq \sqrt{\frac{n}{y}}}
\eqn{\floordiv{\floor{n/y}}{\isqrt{n}+1} \leq \isqrt{n/y},}
as desired.
\end{proof}

\begin{lemma} \label{mfeqljk}
For $1 \leq \chi \leq \sqrt{n}/b$ and large $n$,
\eqn{\frac{bn}{(1+b\chi-b)(b\chi+1)} > b.}
\end{lemma}
\begin{proof}
Since $\sqrt{n}/b > \chi$, we have
\eqn{n > b^2 \chi^2.}
For all but the smallest $n$, we will have $b > 2$; since we are working in the limit of large $n$, we can weaken this to
\eqn{n > b^2 \chi^2 + 2b\chi + 1 - b^2\chi - b}
\eqn{= (1 + b\chi)^2 - b \cdot (b\chi + 1)}
\eqn{= (1 + b\chi - b) (b\chi + 1),}
from which the conclusion follows immediately.
\end{proof}

\begin{lemma} \label{mfeqljk_u}
For $1 + b \leq u \leq 1 + \sqrt{n}$ and large $n$,
\eqn{\frac{bn}{(u-b) \cdot u} > b.}
\end{lemma}
\begin{proof}
Since $\sqrt{n} > u-1$, we have
\eqn{n > (u-1)^2.}
For all but the smallest $n$, we will have $b > 2$; since we are working in the limit of large $n$, we can weaken this to
\eqn{n > (u-1)^2 - (bu - 2u + 1) = u^2 - bu,}
from which the conclusion follows immediately.
\end{proof}

\begin{lemma} \label{ouawrt4coi}
Suppose that $1 \leq t \leq b \leq n^{1/3}$.  Then for sufficiently large $b$ and $n$,
\eqn{\isqrt{n/t} \leq 1 + \floordiv{n}{t \cdot (b+1)}.}
\end{lemma}
\begin{proof}
Consider the function
\eqn{f(z) = \left( 1 + \frac{z}{b+1} \right)^2 - z.}
This is a quadratic in $z$.  We evaluate it at four points: $z \in \set{1, 2, b^2/2, b^2}$.
\eqn{f(1) = \left( 1 + \frac{1}{b+1} \right)^2 - 1 = \frac{2}{b+1} + \frac{1}{(b+1)^2} = \frac{2b + 3}{(b+1)^2} > 0}
%\eqn{= \frac{2(b+1)+1}{(b+1)^2}}
%\eqn{= \frac{2b+3}{(b+1)^2}}
\eqn{f(2) = \left( 1 + \frac{2}{b+1} \right)^2 - 2 = -1 + \frac{4}{b+1} + \frac{4}{(b+1)^2} = \frac{-b^2 + 2b + 7}{(b+1)^2} < 0}
%\eqn{= \frac{-(b+1)^2 + 4(b+1) + 4}{(b+1)^2}}
%\eqn{= \frac{-b^2 - 2b - 1 + 4b + 4 + 4}{(b+1)^2}}
%\eqn{= \frac{-b^2 + 2b + 7}{(b+1)^2}}
\eqn{f(b^2/2) = \left( 1 + \frac{b^2/2}{b+1} \right)^2 - \frac{b^2}{2} = 1 + \frac{b^2}{b+1} + \frac{1}{4} \cdot \frac{b^4}{(b+1)^2} - \frac{b^2}{2} = \frac{-b^4 + 6b^2 + 8b + 4}{4 \cdot (b+1)^2} < 0}
%\eqn{= \frac{4 \cdot (b+1)^2 + 4 \cdot b^2 \cdot (b+1) + b^4 - 2b^2 \cdot (b+1)^2}{4 \cdot (b+1)^2}}
%\eqn{= \frac{(4b^2+8b+4) + (4b^3+4b^2) + b^4 - (2b^4 + 4b^3 + 2b^2)}{4 \cdot (b+1)^2}}
%\eqn{= \frac{(4b^2+8b+4) + (4b^3+4b^2) - b^4 - 4b^3 - 2b^2}{4 \cdot (b+1)^2}}
%\eqn{= \frac{(4b^2+8b+4) - b^4 + 2b^2}{4 \cdot (b+1)^2}}
%\eqn{= \frac{-b^4 + 6b^2 + 8b + 4}{4 \cdot (b+1)^2}}
%\eqn{= \frac{-b^4 + 6b^2 + 8b + 4}{4 \cdot (b+1)^2}}
%\eqn{< 0}
\eqn{f(b^2) = \left(1 + \frac{b^2}{b+1}\right)^2 - b^2 = 1 + \frac{2b^2}{b+1} + \frac{b^4}{(b+1)^2} - b^2 = \frac{2b^2 + 2b + 1}{(b+1)^2} > 0}
%\eqn{= \frac{(b+1)^2 + 2b^2 \cdot (b+1) + b^4 - b^2 \cdot (b+1)^2}{(b+1)^2}}
%\eqn{= \frac{(b^2+2b+1) + (2b^3+2b^2) + b^4 - (b^4+2b^3+b^2)}{(b+1)^2}}
%\eqn{= \frac{2b^2 + 2b + 1}{(b+1)^2}}
%\eqn{> 0}
Therefore, for sufficiently large $b$, if $z \geq b^2$, then $f(z) > 0$.

Now consider $f(n/t)$.  From the hypotheses of this lemma, we have $n/t \geq b^2$.  Therefore,
\eqn{0 < f(n/t)}
\eqn{0 \leq f(n/t)}
\eqn{0 \leq \left( 1 + \frac{n/t}{b+1} \right)^2 - \frac{n}{t}}
\eqn{\frac{n}{t} \leq \left( 1 + \frac{n}{t \cdot (b+1)} \right)^2}
\eqn{\sqrt{n/t} \leq 1 + \frac{n}{t \cdot (b+1)}}
Applying the floor function to both sides then yields the desired result.
\end{proof}

\begin{lemma} \label{okiagfvwr}
Suppose that $2 \leq t$ and $b = o(\sqrt{n})$.  Then for sufficiently large $n$,
\eqn{\isqrt{n/t} \geq \floordiv{n}{t \cdot \left(1 + b \cdot \dfloordiv{\isqrt{n}}{b}\right)}.}
\end{lemma}
\begin{proof}
Since $b = o(n^{1/2})$, we have
\eqn{\isqrt{n} - b \geq \sqrt{\frac{n}{2}}}
for sufficiently large $n$.  This can be weakened to
\eqn{\isqrt{n} - (\isqrt{n} \bmod b) \geq \sqrt{\frac{n}{2}},}
which is equivalent to
\eqn{b \cdot \floordiv{\isqrt{n}}{b} \geq \sqrt{\frac{n}{2}},}
which can be weakened to
\eqn{1 + b \cdot \floordiv{\isqrt{n}}{b} \geq \sqrt{\frac{n}{2}}.}
By hypothesis, $t \geq 2$, so we can weaken this to
\eqn{1 + b \cdot \floordiv{\isqrt{n}}{b} \geq \sqrt{\frac{n}{t}},}
which is equivalent to
\eqn{1 \geq \frac{\sqrt{n/t}}{1 + b \cdot \dfloordiv{\isqrt{n}}{b}}}
\eqn{\sqrt{n/t} \geq \frac{n/t}{1 + b \cdot \dfloordiv{\isqrt{n}}{b}}.}
Applying the floor function to both sides then yields the desired result.
\end{proof}

\begin{lemma} \label{lmmfanb}
$U \leq 2 \sqrt{n}$.
\end{lemma}
\begin{proof}
Since $b = o(\sqrt{n})$, we have for sufficiently large $n$
\eqn{n \leq 2 \cdot \left(1 + \isqrt{n}\right)^2 - 2b \cdot \left(\isqrt{n}+1\right)}
\eqn{ = 2 \cdot \left(1 + \isqrt{n} - b\right)\left(\isqrt{n}+1\right)}
\eqn{ \leq 2 \cdot \left(1 + \isqrt{n} - (\isqrt{n} \bmod b)\right)\left(\isqrt{n}+1\right)}
\eqn{ = 2 \cdot \left(1 + b \cdot \floordiv{\isqrt{n}}{b}\right)\left(\isqrt{n}+1\right),}
which is equivalent to
\eqn{2 \sqrt{n} \geq n \cdot \frac{\sqrt{n}}{\left(1 + b \cdot \dfloordiv{\isqrt{n}}{b}\right)\left(\isqrt{n}+1\right)}}
\eqn{ \geq n \cdot \frac{\isqrt{n}}{\left(1 + b \cdot \dfloordiv{\isqrt{n}}{b}\right)\left(\isqrt{n}+1\right)}}
\eqn{ \geq n \cdot \frac{\isqrt{n} - b \cdot \dfloordiv{\isqrt{n}}{b}}{\left(1 + b \cdot \dfloordiv{\isqrt{n}}{b}\right)\left(\isqrt{n}+1\right)},}
which is the desired result.
\end{proof}

\begin{lemma} \label{catalan3}
For $k \geq 2$, $C_k > 3^k / 6$.
\end{lemma}
\begin{proof}
We proceed by induction.  The base cases are $2 \leq k \leq 4$.  For the inductive step, we have
\eqn{C_{k+1} = C_k \cdot \frac{C_{k+1}}{C_k} > \frac{3^k}{6} \cdot \frac{C_{k+1}}{C_k} = \frac{3^k}{6} \cdot 2 \cdot \frac{2k+1}{k+2}.}
The quantity $(2k+1)/(k+2)$ equals $3/2$ at $k=4$, and increases from there.  Therefore
\eqn{\geq \frac{3^k}{6} \cdot 3 = \frac{3^{k+1}}{6},}
which completes the induction.
\end{proof}


\begin{lemma} \label{BigIntegral}
\eqn{\sum_{\substack{2b \leq x \leq \sqrt{n} \\ b \mid x}}\sum_{t=1}^b \left( 1 + \max\left( \min \left( \isqrt{n/t} , \floordiv{n}{t \cdot (1 + x - b)} \right)
-
\left( 1 + \floordiv{n}{t \cdot (x+1)} \right)
, 0 \right) \right)}
\eqn{\approx \frac{n}{b} \cdot \left( \ln(4) + \ln(2)^2 - \ln(2) \ln(n) + \ln(8)\ln(b) \right)}
\end{lemma}
\begin{proof}
Reindexing the outer sum yields
\eqn{\sum_{\chi=2}^{\sqrt{n}/b} \sum_{t=1}^b \left( 1 + \max\left( \min \left( \isqrt{n/t} , \floordiv{n}{t \cdot (1 + b\chi - b)} \right)
- 1 - \floordiv{n}{t \cdot (b\chi + 1)}
, 0 \right) \right)}

\todo b-n-chi / bianchi pun

\eqn{= \sqrt{n} + \sum_{\chi=2}^{\sqrt{n}/b} \sum_{t=1}^b \max\left( \min \left( \isqrt{n/t} , \floordiv{n}{t \cdot (1 + b\chi - b)} \right)
- 1 - \floordiv{n}{t \cdot (b\chi + 1)}
, 0 \right)}

The $\sqrt{n}$ is neglectably small.

\eqn{\approx \int_2^{\sqrt{n}/b} \int_1^b
\max\left( \min \left( \sqrt{n/t} , \frac{n}{t \cdot (1 + b\chi - b)} \right)
- 1 - \frac{n}{t \cdot (b\chi + 1)}
, 0 \right) dt \; d\chi}

Let $u=b\chi+1$.

\eqn{= \frac{1}{b} \cdot \int_{1+2b}^{1+\sqrt{n}} \int_1^b
\max\left( \min \left( \sqrt{n/t} , \frac{n}{t \cdot (u - b)} \right) - 1 - \frac{n}{tu} , 0 \right)
dt \; du}

Let $T \defeq n \cdot (u-b)^{-2}$.  This is the crossover point in the $\min$ function.  For lesser $t$, the first argument prevails, and the integral becomes

\eqn{= \frac{1}{b} \cdot \int_{1+2b}^{1+\sqrt{n}} \left(
\int_T^b \max\left( \frac{n}{t \cdot (u - b)} - 1 - \frac{n}{tu} , 0 \right) dt
+ \int_1^T \max\left( \sqrt{n/t} - 1 - \frac{n}{tu} , 0 \right) dt
\right) du
}

\eqn{= \frac{1}{b} \cdot \int_{1+2b}^{1+\sqrt{n}} \left(
\int_T^b \max\left( \frac{bn}{tu \cdot (u - b)} - 1 , 0 \right) dt
+ \int_1^T \max\left( \sqrt{n/t} - 1 - \frac{n}{tu} , 0 \right) dt
\right) du
}

By Lemma \ref{mfeqljk_u}, the $\max( ,0)$ can be dropped from the first integral.

\eqn{= \frac{1}{b} \cdot \int_{1+2b}^{1+\sqrt{n}} \left(
\int_T^b \left( \frac{bn}{tu \cdot (u - b)} - 1 \right) dt
+ \int_1^T \max\left( \sqrt{n/t} - 1 - \frac{n}{tu} , 0 \right) dt
\right) du
}

Let $\displaystyle S \defeq \frac{n}{2} - \frac{n}{u} - \sqrt{\left(\frac{n}{2}\right)^2 - \frac{n^2}{u}}$.  This is the crossover point in the remaining $\max$ function; the first argument prevails for $t > S$.

\eqn{= \frac{1}{b} \cdot \int_{1+2b}^{1+\sqrt{n}} \left(
\int_T^b \left( \frac{bn}{tu \cdot (u - b)} - 1 \right) dt
+ \int_S^T \left( \sqrt{n/t} - 1 - \frac{n}{tu} \right) dt
\right) du
}

\eqn{= \frac{1}{b} \cdot \int_{1+2b}^{1+\sqrt{n}} \left(
\frac{bn \ln(b/T)}{u \cdot (u - b)} - b + T + 2\sqrt{n}(T^{1/2} - S^{1/2}) - (T - S) - \frac{n \ln(T/S)}{u}
\right) du
}

\eqn{= \frac{1}{b} \cdot \int_{1+2b}^{1+\sqrt{n}} \left(
\frac{bn \ln(b/T)}{(u - b) \cdot u} - b + 2\sqrt{nT} - 2\sqrt{nS} + S + \frac{n \ln(S/T)}{u}
\right) du
}

\eqn{= \frac{1}{b} \cdot \int_{1+2b}^{1+\sqrt{n}} \left(
\frac{bn \ln\left(\dfrac{b}{n}(u-b)^2\right)}{(u - b) \cdot u} - b + \frac{2n}{u-b} - 2n\sqrt{\frac{S}{n}} + S + \frac{n \ln\left(\dfrac{S}{n}(u-b)^2\right)}{u}
\right) du
}

Observe that $\displaystyle \sqrt{\frac{S}{n}} = \frac{S}{n} + \frac{1}{u}$.

\eqn{= \frac{1}{b} \cdot \int_{1+2b}^{1+\sqrt{n}} \left(
\frac{bn \ln\left(\dfrac{b}{n}(u-b)^2\right)}{(u - b) \cdot u} - b + \frac{2n}{u-b} - 2n\cdot\left(\frac{S}{n}+\frac{1}{u}\right) + S + \frac{n \ln\left(\dfrac{S}{n}(u-b)^2\right)}{u}
\right) du
}

\eqn{= \frac{n}{b} \cdot \int_{1+2b}^{1+\sqrt{n}} \left(
\frac{b \ln\left(\dfrac{b}{n}(u-b)^2\right)}{(u - b) \cdot u} - \frac{b}{n} + \frac{2}{u-b} - \frac{2}{u} - \frac{S}{n} + \frac{\ln\left(\dfrac{S}{n}(u-b)^2\right)}{u}
\right) du
}

\eqn{= 2b - \sqrt{n} + \frac{n}{b} \cdot \int_{1+2b}^{1+\sqrt{n}} \left(
\frac{b \ln\left(\dfrac{b}{n}(u-b)^2\right)}{(u - b) \cdot u} + \frac{2}{u-b} - \frac{2}{u} - \frac{S}{n} + \frac{\ln\left(\dfrac{S}{n}(u-b)^2\right)}{u}
\right) du
}

By Lemma \ref{Snu5int},

\eqn{= 2b - \sqrt{n} - \frac{n}{b} \cdot O(b^{-1}) + \frac{n}{b} \cdot \int_{1+2b}^{1+\sqrt{n}} \left(
\frac{b \ln\left(\dfrac{b}{n}(u-b)^2\right)}{(u - b) \cdot u} + \frac{2}{u-b} - \frac{2}{u} + \frac{\ln\left(\dfrac{S}{n}(u-b)^2\right)}{u}
\right) du
}

The terms that have been pulled out are neglectably small.

\eqn{\approx \frac{n}{b} \cdot \int_{1+2b}^{1+\sqrt{n}} \left(
\frac{b \ln\left(\dfrac{b}{n}(u-b)^2\right)}{(u - b) \cdot u} + \frac{2}{u-b} - \frac{2}{u} + \frac{\ln\left(\dfrac{S}{n}(u-b)^2\right)}{u}
\right) du
}

\eqn{= \frac{n}{b} \cdot \int_{1+2b}^{1+\sqrt{n}} \left(
\frac{b \ln\left(\dfrac{b}{n}(u-b)^2\right)}{(u - b) \cdot u} + \frac{2}{u-b} - \frac{2}{u} + \frac{\ln\left(\dfrac{S}{n} \cdot u^2 \cdot \dfrac{(u-b)^2}{u^2}\right)}{u}
\right) du
}

\eqn{= \frac{n}{b} \cdot \int_{1+2b}^{1+\sqrt{n}} \left(
\frac{b \ln\left(\dfrac{b}{n}(u-b)^2\right)}{(u - b) \cdot u} + \frac{2}{u-b} - \frac{2}{u} + \frac{2\ln(u-b)}{u} - \frac{2\ln(u)}{u} + \frac{\ln\left(\dfrac{S}{n}\cdot u^2\right)}{u}
\right) du
}

By Lemma \ref{keysmash},

\eqn{= \frac{n}{b} \cdot \int_{1+2b}^{1+\sqrt{n}} \left(
\frac{b \ln\left(\dfrac{b}{n}(u-b)^2\right)}{(u - b) \cdot u} + \frac{2}{u-b} - \frac{2}{u} + \frac{2\ln(u-b)}{u} - \frac{2\ln(u)}{u}
\right) du + \frac{n}{b} \cdot O(b^{-1})
}

The term that has been pulled out is neglectably small.

\eqn{\approx \frac{n}{b} \cdot \int_{1+2b}^{1+\sqrt{n}} \left(
\frac{b \ln\left(\dfrac{b}{n}(u-b)^2\right)}{(u - b) \cdot u} + \frac{2}{u-b} - \frac{2}{u} + \frac{2\ln(u-b)}{u} - \frac{2\ln(u)}{u}
\right) du
}

\eqn{= \frac{n}{b} \cdot \int_{1+2b}^{1+\sqrt{n}} \left(
\left(\ln\left(\frac{b}{n}\right) + \ln(u-b)^2\right) \left( \frac{1}{u-b} - \frac{1}{u} \right) + \frac{2}{u-b} - \frac{2}{u} + \frac{2\ln(u-b)}{u} - \frac{2\ln(u)}{u}
\right) du
}

\eqn{= \frac{n}{b} \cdot \int_{1+2b}^{1+\sqrt{n}} \left(
\frac{\ln(b/n)}{u-b} - \frac{\ln(b/n)}{u} + \frac{2\ln(u-b)}{u-b} + \frac{2}{u-b} - \frac{2}{u} - \frac{2\ln(u)}{u}
\right) du
}

\eqn{= \frac{n}{b} \cdot \int_{1+2b}^{1+\sqrt{n}} \left( \frac{\ln(b/n)+2}{u-b} - \frac{\ln(b/n)+2}{u} + \frac{2\ln(u-b)}{u-b} - \frac{2\ln(u)}{u} \right) du }
\eqn{= \frac{n}{b} \cdot \int_{1+2b}^{1+\sqrt{n}} \left( \left(2+\ln\left(\frac{b}{n}\right)\right)\left(\frac{1}{u-b}-\frac{1}{u}\right) + \frac{2\ln(u-b)}{u-b} - \frac{2\ln(u)}{u} \right) du }
\eqn{= \frac{n}{b} \cdot \eval{\left( \left(2+\ln\left(\frac{b}{n}\right)\right)\left(\ln(u-b)-\ln(u)\right)+(\ln(u-b))^2-(\ln(u))^2 \right)}{u=1+2b}{1+\sqrt{n}}}
\eqn{= \frac{n}{b} \cdot \eval{\left( \left(2+\ln\left(\frac{b}{n}\right)\right) \ln\left(1-\frac{b}{u}\right) + (\ln(u-b)+\ln(u))(\ln(u-b)-\ln(u)) \right)}{u=1+2b}{1+\sqrt{n}}}
\eqn{= \frac{n}{b} \cdot \eval{\left( \left(2+\ln\left(\frac{b}{n}\right)\right) \ln\left(1-\frac{b}{u}\right) + \ln(u^2-bu)\ln\left(1-\frac{b}{u}\right) \right)}{u=1+2b}{1+\sqrt{n}}}
\eqn{= \frac{n}{b} \cdot \eval{\left( \left(2+\ln\left(\frac{b}{n}\right)+\ln(u^2-bu)\right) \ln\left(1-\frac{b}{u}\right) \right)}{u=1+2b}{1+\sqrt{n}}}
\eqn{\approx \frac{n}{b} \cdot \eval{\left( \left(2+\ln\left(\frac{b}{n}\right)+\ln(u^2-bu)\right) \ln\left(1-\frac{b}{u}\right) \right)}{u=2b}{\sqrt{n}}}
\eqn{= \frac{n}{b} \cdot \left( \left(2+\ln\left(\frac{b}{n}\right)+\ln(n-b\sqrt{n})\right) \ln\left(1-\frac{b}{\sqrt{n}}\right) - \left(2+\ln\left(\frac{b}{n}\right)+\ln(2b^2)\right) \ln\left(\frac{1}{2}\right) \right)}
\eqn{\approx \frac{n}{b} \cdot \left( \left(2+\ln\left(\frac{b}{n}\right)+\ln(n)\right) \cdot \frac{-b}{\sqrt{n}} - \left(2+\ln\left(\frac{b}{n}\right)+\ln(2b^2)\right) \ln\left(\frac{1}{2}\right) \right)}
The first term is neglectably small.
\eqn{\approx \frac{n}{b} \cdot \left( \left(2+\ln\left(\frac{b}{n}\right)+\ln(2b^2)\right) \ln(2) \right)}
\eqn{= \frac{n}{b} \cdot \left( \ln(4) + \ln(2)^2 - \ln(2)\ln(n) + \ln(8)\ln(b) \right),}
as desired.
\end{proof}

\begin{lemma} \label{keysmash}
\eqn{0 < \int_{1+2b}^{1+\sqrt{n}} \left( \frac{1}{u} \cdot \ln \left( u^2 \cdot \frac{S}{n} \right) \right) du < \frac{17}{b}.}
\end{lemma}
\begin{proof}
Let $C_k$ be the $k$\textsuperscript{th} Catalan number (\oeisref{000108}).  Observe that
\eqn{\frac{S}{n} = \sum_{k=2}^\infty \frac{C_{k-1}}{u^k},}
and therefore $S/n > u^{-2}$.  Then
\eqn{0 < \int_{1+2b}^{1+\sqrt{n}} \left( \frac{1}{u} \cdot \ln \left( u^2 \cdot \frac{S}{n} \right) \right) du < \int_{1+2b}^{1+\sqrt{n}} \left( \frac{1}{u} \cdot \ln \left( u^2 \cdot \sum_{k=2}^\infty \frac{C_{k-1}}{u^k}\right) \right) du }
\eqn{ = \int_{1+2b}^{1+\sqrt{n}} \left( \frac{1}{u} \cdot \ln \left( \sum_{k=2}^\infty \frac{C_{k-1}}{u^{k-2}}\right) \right) du }
\eqn{ = \int_{1+2b}^{1+\sqrt{n}} \left( \frac{1}{u} \cdot \ln \left( \sum_{k=0}^\infty \frac{C_{k+1}}{u^k}\right) \right) du }
%\eqn{ = \int_{1+2b}^{1+\sqrt{n}} \left( \frac{1}{u} \cdot \ln \left( C_1 + \sum_{k=1}^\infty \frac{C_{k+1}}{u^k}\right) \right) du }
\eqn{ = \int_{1+2b}^{1+\sqrt{n}} \left( \frac{1}{u} \cdot \ln \left( 1 + \sum_{k=1}^\infty \frac{C_{k+1}}{u^k}\right) \right) du }
\eqn{ < \int_{1+2b}^{1+\sqrt{n}} \left( \frac{1}{u} \cdot \ln \left( 1 + \sum_{k=1}^\infty \frac{4^{k+1}}{u^k}\right) \right) du }
%\eqn{ = \int_{1+2b}^{1+\sqrt{n}} \left( \frac{1}{u} \cdot \ln \left( 1 + 4 \sum_{k=1}^\infty (4/u)^k \right) \right) du }
%\eqn{ = \int_{1+2b}^{1+\sqrt{n}} \left( \frac{1}{u} \cdot \ln \left( 1 + 4 \cdot \frac{4/u}{1-4/u} \right) \right) du }
\eqn{ = \int_{1+2b}^{1+\sqrt{n}} \left( \frac{1}{u} \cdot \ln \left( 1 + \frac{16}{u-4} \right) \right) du }
\eqn{ = \int_{1+2b}^{1+\sqrt{n}} \left( \frac{1}{u} \cdot \sum_{k=1}^\infty \left( \frac{(-1)^{k+1}}{k} \cdot \left(\frac{16}{u-4}\right)^k \right) \right) du }
%\eqn{ = \int_{1+2b}^{1+\sqrt{n}} \sum_{k=1}^\infty \left( \frac{(-1)^{k+1}}{ku} \cdot \left(\frac{16}{u-4}\right)^k \right) du }
\eqn{ = \int_{1+2b}^{1+\sqrt{n}} \sum_{k=1}^\infty \left( \frac{(-16)^{k}}{-k} \cdot \frac{1}{u} \cdot \frac{1}{(u-4)^k} \right) du }
We have absolute convergence, so
\eqn{ = \sum_{k=1}^\infty \frac{(-16)^k}{-k} \cdot \int_{1+2b}^{1+\sqrt{n}} \left( \frac{1}{u} \cdot \frac{1}{(u-4)^k} \right) du }
\eqn{ < \sum_{k=1}^\infty \frac{16^k}{k} \cdot \int_{1+2b}^{1+\sqrt{n}} \left( \frac{1}{u} \cdot \frac{1}{(u/2)^k} \right) du }
\eqn{ = \sum_{k=1}^\infty \frac{32^k}{k} \cdot \int_{1+2b}^{1+\sqrt{n}} u^{-k-1} \; du }
\eqn{ = \sum_{k=1}^\infty \frac{32^k}{-k^2} \cdot \eval{u^{-k}}{u=1+2b}{1+\sqrt{n}} }
%\eqn{ = \sum_{k=1}^\infty \frac{32^k}{-k^2} \cdot \left(\frac{1}{(1+\sqrt{n})^k} - \frac{1}{(1+2b)^k}\right) }
\eqn{ = \sum_{k=1}^\infty \frac{32^k}{k^2} \cdot \left(\frac{1}{(1+2b)^k} - \frac{1}{(1+\sqrt{n})^k}\right) }
\eqn{ < \sum_{k=1}^\infty \left(\frac{32}{1+2b}\right)^k}
\eqn{ = \frac{32/(1+2b)}{1-32/(1+2b)}}
\eqn{ = \frac{32}{2b-31}}
\eqn{ < \frac{17}{b},}
as desired.
\end{proof}

\begin{lemma} \label{Snu5}
For $u > 5$, $\disp 0 < \frac{1}{2} - \frac{1}{u} - \sqrt{\frac{1}{4} - \frac{1}{u}} < \frac{2}{u^2}$.
\end{lemma}
\begin{proof}
\eqn{\frac{16}{u^2} + \frac{16}{u} < 4}
\eqn{u^2 + 4 + \frac{16}{u^2} - 4u - 8 + \frac{16}{u} < u^2 - 4u}
\eqn{u - 2 - \frac{4}{u} < \sqrt{u^2 - 4u}}
\eqn{u - 2 - \sqrt{u^2 - 4u} < \frac{4}{u}}
\eqn{0 < u - 2 - \sqrt{u^2 - 4u} < \frac{4}{u}}
\eqn{0 < \frac{1}{2} - \frac{1}{u} - \sqrt{\frac{1}{4} - \frac{1}{u}} < \frac{2}{u^2}}
\end{proof}

\begin{lemma} \label{Snu5int}
$\disp 0 < \int_{1+2b}^{1+\sqrt{n}} \frac{S}{n} \; du < \frac{2}{1+2b} - \frac{2}{1+\sqrt{n}}$.
\end{lemma}
\begin{proof}
For $u > 5$, we have
\eqn{\frac{16}{u^2} + \frac{16}{u} < 4}
\eqn{u^2 + 4 + \frac{16}{u^2} - 4u - 8 + \frac{16}{u} < u^2 - 4u}
\eqn{u - 2 - \frac{4}{u} < \sqrt{u^2 - 4u}}
\eqn{u - 2 - \sqrt{u^2 - 4u} < \frac{4}{u}}
\eqn{0 < u - 2 - \sqrt{u^2 - 4u} < \frac{4}{u}}
\eqn{0 < \frac{1}{2} - \frac{1}{u} - \sqrt{\frac{1}{4} - \frac{1}{u}} < \frac{2}{u^2}}
\eqn{0 < \frac{S}{n} < \frac{2}{u^2}}
Integrating then gives the desired result.
\end{proof}

\begin{lemma} \label{sqrtapprox}
$\disp \lim_{u\rightarrow\infty} u^4 \cdot \left( \frac{1}{2} - \frac{1}{u} - \sqrt{\frac{1}{4}-\frac{1}{u}} - \frac{1}{u\cdot(u-2)} \right) = 1$.
\end{lemma}
\begin{proof}
Make the substitution $u=1/x$ to obtain
\eqn{\lim_{x\rightarrow0^+} x^{-4} \cdot \left( \frac{1}{2} - x - \sqrt{\frac{1}{4}-x} - \frac{1}{x^{-1}\cdot(x^{-1}-2)} \right)}
%\eqn{ = \lim_{x\rightarrow0^+} \frac{\dfrac{1}{2} - x - \sqrt{\dfrac{1}{4}-x} - \dfrac{x^2}{1-2x}}{x^4}}
%\eqn{ = \lim_{x\rightarrow0^+} \frac{1 - 2x - \sqrt{1-4x} - \dfrac{2x^2}{1-2x}}{2x^4}}
%\eqn{ = \lim_{x\rightarrow0^+} \frac{1 - 2x - \sqrt{1-4x} + \left(x + \frac{1}{2} - \frac{1/2}{1-2x}\right)}{2x^4}}
%\eqn{ = \lim_{x\rightarrow0^+} \frac{\frac{3}{2} - x - \sqrt{1-4x} + \frac{1/2}{2x-1}}{2x^4}}
\eqn{ = \lim_{x\rightarrow0^+} \frac{3 - 2x - 2\sqrt{1-4x} + (2x-1)^{-1}}{4x^4}}
L'H\^{o}pital's rule yields
\eqn{ = \lim_{x\rightarrow0^+} \frac{0 - 2 - 2\dfrac{-4}{2\sqrt{1-4x}} + (-1)(2x-1)^{-2}(2)}{16x^3}}
\eqn{ = \lim_{x\rightarrow0^+} \frac{-2 + 4(1-4x)^{-1/2} - 2(2x-1)^{-2}}{16x^3}}
L'H\^{o}pital's rule yields
\eqn{ = \lim_{x\rightarrow0^+} \frac{0 + 4(-1/2)(1-4x)^{-3/2}(-4) - 2(-2)(2x-1)^{-3}(2)}{48x^2}}
\eqn{ = \lim_{x\rightarrow0^+} \frac{8(1-4x)^{-3/2} + 8(2x-1)^{-3}}{48x^2}}
L'H\^{o}pital's rule yields
\eqn{ = \lim_{x\rightarrow0^+} \frac{8(-3/2)(1-4x)^{-5/2}(-4) + 8(-3)(2x-1)^{-4}(2)}{96x}}
\eqn{ = \lim_{x\rightarrow0^+} \frac{48(1-4x)^{-5/2} - 48(2x-1)^{-4}}{96x}}
L'H\^{o}pital's rule yields
\eqn{ = \lim_{x\rightarrow0^+} \frac{48(-5/2)(1-4x)^{-7/2}(-4) - 48(-4)(2x-1)^{-5}(2)}{96}}
\eqn{ = \lim_{x\rightarrow0^+} \frac{480(1-4x)^{-7/2} + 384(2x-1)^{-5}}{96}}
\eqn{ = \frac{480(1-4\cdot0)^{-7/2} + 384(2\cdot0-1)^{-5}}{96}}
\eqn{ = 1}
\end{proof}

\begin{lemma} \label{sqrtapprox2}
$\disp \lim_{u\rightarrow\infty} u^3 \cdot \left( \frac{1}{2} - \frac{1}{u} - \sqrt{\frac{1}{4}-\frac{1}{u}} - \frac{1}{u^2} \right) = 2$.
\end{lemma}
\begin{proof}
Make the substitution $u=1/x$ to obtain
\eqn{\lim_{x\rightarrow0^+} x^{-3} \cdot \left( \frac{1}{2} - x - \sqrt{\frac{1}{4}-x} - x^2 \right)}
%\eqn{ = \lim_{x\rightarrow0^+} \frac{\dfrac{1}{2} - x - \sqrt{\dfrac{1}{4}-x} - x^2}{x^3}}
\eqn{ = \lim_{x\rightarrow0^+} \frac{1 - 2x - (1-4x)^{1/2} - 2x^2}{2x^3}}
L'H\^{o}pital's rule yields
\eqn{ = \lim_{x\rightarrow0^+} \frac{0 - 2 - (1/2)(1-4x)^{-1/2}(-4) - 4x}{6x^2}}
\eqn{ = \lim_{x\rightarrow0^+} \frac{-2 + 2(1-4x)^{-1/2} - 4x}{6x^2}}
L'H\^{o}pital's rule yields
\eqn{ = \lim_{x\rightarrow0^+} \frac{0 + 2(-1/2)(1-4x)^{-3/2}(-4) - 4}{12x}}
\eqn{ = \lim_{x\rightarrow0^+} \frac{4(1-4x)^{-3/2} - 4}{12x}}
L'H\^{o}pital's rule yields
\eqn{ = \lim_{x\rightarrow0^+} \frac{4(-3/2)(1-4x)^{-5/2}(-4) - 0}{12}}
\eqn{ = \lim_{x\rightarrow0^+} \frac{24(1-4x)^{-5/2}}{12}}
\eqn{ = 2}
\end{proof}

\nocite{*}

\oeisref{002088}
\oeisref{064018}

\setlength{\bibitemsep}{\parskip}
\printbibliography[heading=bibnumbered]

\end{document}



