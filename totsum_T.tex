\documentclass{article}
\renewcommand*\rmdefault{ppl}
\usepackage[margin=2cm]{geometry} \geometry{letterpaper}
\usepackage[parfill]{parskip}
\usepackage[utf8]{inputenc}
\usepackage[english]{babel}
\usepackage[style=numeric,backref=true,backrefstyle=none,abbreviate=false,urldate=iso,seconds=true]{biblatex} \addbibresource{}
\usepackage{csquotes}
\usepackage{amssymb,amsmath,amsthm}
\usepackage{graphicx}
\usepackage{tikz}
\usepackage{tikzscale}
\usepackage{pgfplots} \pgfplotsset{compat=1.15}
\usepackage{url}
\usepackage{multicol}
\usepackage[yyyymmdd,hhmmss]{datetime}
\usepackage[bottom]{footmisc}
\usepackage{hyperref}
\usepackage{color} \newcommand{\todo}[0]{{\textcolor{red}{\textbf{TODO: }}}}
\usepackage{enumitem}
\DeclareGraphicsRule{.tif}{png}{.png}{`convert #1 `dirname #1`/`basename #1 .tif`.png}
\newcommand{\eqn}[1]{\begin{displaymath} #1 \end{displaymath}}
\newcommand{\neqn}[1]{\begin{equation} #1 \end{equation}}
\newcommand{\tweqn}[1]{\begin{displaymath}\resizebox{\textwidth}{!}{$\displaystyle #1 $}\end{displaymath}}
\newcommand{\ntweqn}[1]{\begin{equation}\resizebox{\textwidth}{!}{$\displaystyle #1 $}\end{equation}}
\newcommand{\floor}[1]{{\left\lfloor #1 \right\rfloor}}
\newcommand{\ceil}[1]{{\left\lceil #1 \right\rceil}}
\newcommand{\vect}[1]{\left\langle #1 \right\rangle}
\newcommand{\deriv}[2]{\frac{d #1}{d #2}}
\newcommand{\derop}[1]{\frac{d}{d #1}}
\newcommand{\partiald}[2]{\frac{\partial #1}{\partial #2}}
\newcommand{\partialop}[1]{\frac{\partial}{\partial #1}}
\newcommand{\integral}[4]{\displaystyle\int_{#3}^{#4} \! #1 \, d#2}
\newcommand{\dintegral}[4]{\displaystyle\int\!\!\!\!\int_{#3}^{#4} #1 \, d#2}
\newcommand{\tintegral}[4]{\displaystyle\int\!\!\!\!\int\!\!\!\!\int_{#3}^{#4} #1 \, d#2}
\newcommand{\disp}[0]{\displaystyle}
\newcommand{\abs}[1]{\left\vert #1 \right\vert}
\newcommand{\grad}[0]{\vec{\nabla\!}\,}
\newcommand{\epsil}[0]{\varepsilon}
\newcommand{\eval}[3]{\left. #1 \right|_{#2}^{#3}}
\newcommand{\lag}[0]{\mathcal{L}}
\newcommand{\ham}[0]{\mathcal{H}}
\newcommand{\realpart}[1]{{\mathfrak{Re}\!\left\{#1\right\}}}
\newcommand{\ipart}[1]{{\mathfrak{Im}\!\left\{#1\right\}}}
\newcommand{\set}[1]{{\left\{#1\right\}}}
\newcommand{\lcm}[0]{\operatorname{lcm}}
\newcommand{\defeq}[0]{\overset{\mathrm{def}}{=}}
\newcommand{\oeis}[0]{$^{\texttt{OE}}_{\texttt{IS}}$}
\newcommand{\oeisref}[1]{$^{\texttt{OE}}_{\texttt{IS}}$~\href{https://oeis.org/#1}{#1}}
\newcommand{\quadtext}[1]{\quad \text{#1} \quad}
\newcommand{\qquadtext}[1]{\qquad \text{#1} \qquad}

\allowdisplaybreaks[4]

\def\grabtimezone #1#2#3#4#5#6#7#8#9{\grabtimezoneB}
\def\grabtimezoneB #1#2#3#4#5#6#7{\grabtimezoneC}
\def\grabtimezoneC #1#2'#3'{$#1$#2:#3}
\newcommand{\timezone}[0]{UTC\expandafter \grabtimezone\pdfcreationdate}
\newcommand{\currentdate}{\the\year--\twodigit{\the\month}--\twodigit{\the\day}}
\newcommand{\currentdatetime}{\currentdate\ / \currenttime}%\ \timezone}

\title{}
\author{Lucas A. Brown}
\date{\currentdatetime}

\usepackage{fancyhdr}
\usepackage{lastpage}
\pagestyle{fancy}
\fancyhf{}
\lhead{} \chead{} \rhead{}
\lfoot{\currentdatetime} \cfoot{} \rfoot{Page \thepage\ of \pageref{LastPage}}
\renewcommand{\headrulewidth}{1pt}
\renewcommand{\footrulewidth}{1pt}

\begingroup
    \makeatletter
    \@for\theoremstyle:=definition,remark,plain\do{%
        \expandafter\g@addto@macro\csname th@\theoremstyle\endcsname{%
            \addtolength\thm@preskip\parskip
            }%
        }
\endgroup
\makeatletter
\renewenvironment{proof}[1][\proofname]{\par
  \vspace{-\topsep}% remove the space after the theorem
  \pushQED{\qed}%
  \normalfont
  \topsep0pt \partopsep0pt % no space before
  \trivlist
  \item[\hskip\labelsep
        \itshape
    #1\@addpunct{.}]\ignorespaces
}{%
  \popQED\endtrivlist\@endpefalse
  \addvspace{0pt} % some space after
}
\makeatother
\makeatletter
\newenvironment{solution}[1][\proofname]{\par
  \vspace{-\topsep}% remove the space after the theorem
  \pushQED{\qed}%
  \normalfont
  \topsep0pt \partopsep0pt % no space before
  \trivlist
  \item[\hskip\labelsep
        \bfseries
    Solution #1\@addpunct{.}]\ignorespaces
}{%
  \popQED\endtrivlist\@endpefalse
  \addvspace{0pt} % some space after
}
\makeatother
\usepackage{thmtools}
\declaretheorem[style=plain]{theorem}
\declaretheorem[sibling=theorem,style=plain]{corollary}
\declaretheorem[sibling=theorem,style=plain]{lemma}
\declaretheorem[sibling=theorem,style=plain]{proposition}
\declaretheorem[sibling=theorem,style=plain]{conjecture}
\declaretheorem[sibling=theorem,style=definition,qed=$\clubsuit$]{definition}
\declaretheorem[sibling=theorem,style=definition,qed=$\clubsuit$]{observation}
\declaretheorem[sibling=theorem,style=definition,qed=$\clubsuit$]{fact}
\declaretheorem[sibling=theorem,style=definition,qed=$\spadesuit$]{example}
\declaretheorem[sibling=theorem,style=definition]{notation}
\declaretheorem[sibling=theorem,style=definition]{question}
\declaretheorem[sibling=theorem,style=remark]{remark}
\declaretheorem[style=definition]{problem}

\renewcommand\qedsymbol{$\blacksquare$}

\hypersetup{
    pdftitle={},
    pdfauthor={Lucas A. Brown},
    pdfsubject={},
    pdfkeywords={},
    colorlinks=true,
    linkcolor=blue,
    urlcolor=blue,
    citecolor=blue,
    %hidelinks,
}

\begin{document}
\maketitle %\thispagestyle{fancy}

\begin{abstract}

\end{abstract}

\setlength{\bibitemsep}{\parskip}
\printbibliography[heading=bibnumbered]

\subsection{The totient-only algorithm}

Another starting point is the Dirichlet convolution $\phi * 1 = I$.  Let $ab=n$; the Dirichlet hyperbola method then yields
\eqn{\sum_{k=1}^n I(k) = \sum_{x=1}^a \sum_{y=1}^{n/x} \phi(x) \cdot 1 + \sum_{y=1}^b \sum_{x=1}^{n/y} \phi(x) \cdot 1 - \sum_{x=1}^a \sum_{y=1}^b \phi(x) \cdot 1}
\eqn{\sum_{k=1}^n k = \sum_{x=1}^a \sum_{y=1}^{n/x} \phi(x) + \sum_{y=1}^b \sum_{x=1}^{n/y} \phi(x) - \sum_{x=1}^a \sum_{y=1}^b \phi(x)}
\eqn{\frac{n \cdot (n+1)}{2} = \sum_{x=1}^a \phi(x) \cdot \floordiv{n}{x} + \sum_{y=1}^b \Phi\left(\floordiv{n}{y}\right) - \sum_{x=1}^a \phi(x) \cdot b}
\eqn{\frac{n \cdot (n+1)}{2} = \sum_{x=1}^a \phi(x) \cdot \floordiv{n}{x} + \Phi(n) + \sum_{y=2}^b \Phi\left(\floordiv{n}{y}\right) - b \cdot \Phi(a)}
\eqn{\Phi(n) = \underbrace{\frac{n \cdot (n+1)}{2} + b \cdot \Phi(a)}_Z - \underbrace{\sum_{x=1}^a \phi(x) \cdot \floordiv{n}{x}}_X - \underbrace{\sum_{y=2}^b \Phi\left(\floordiv{n}{y}\right)}_Y}
This yields the algorithm

\begin{algorithm}[H] \label{AlgoT1}
\DontPrintSemicolon
\caption{Compute $\Phi(n)$ in $\softTheta(n^{2/3})$ time and $\softTheta(n^{1/2})$ space using the totient-only algorithm \cite{griff2023}.  \texttt{totientsumT\_A}}
\KwData{$n \geq 1$}
\KwResult{$\Phi(n)$}
\Begin{
    $a \gets \softTheta(n^{2/3})$; $b \gets \floor{n/a}$; $X \gets 0$; $Y \gets 0$; $Z \gets 0$; $P \gets 0$; $s \gets \isqrt{n}$
    
    \lIf{$\isqrt{n} = \floor{n/\isqrt{n}}$}{$s \gets s-1$}
    
    $\chi \gets \floor{n/s}$
    
    Prepare a segmented Sieve of Eratosthenes to compute $\phi(k)$ for $1 \leq k \leq a$.
    
    Let $\phi$ and $\Phi$ be arrays indexed from $1$ through $\isqrt{n}$, inclusive.
    
    Let $\Phi^\prime$ be an array indexed from $1$ through $\floor{n/\isqrt{n}}$, inclusive, initialized to all zeros.
    
    \For{$x=1$ \KwTo $a$}{
        $P \gets P + \phi(x)$
        
        $X \gets X + \phi(x) \cdot \floor{n/x}$
        
        \If{$x \leq \isqrt{n}$}{
            $\phi_x \gets \phi(x)$
            
            $\Phi_x \gets P$
        }
        
        \If{$x = \chi$}{
            \lIf{$\floor{n/x} \neq b$}{$\Phi^\prime_{\floor{n/x}} \gets P$}
            
            $s \gets s - 1$
            
            $\chi \gets \floor{n/s}$
        }
        
        \lIf{$x \gets a$}{$Z \gets bP + \dfrac{n \cdot (n+1)}{2}$}
    }
    
    \For{$y=b$ \KwTo $2$}{
        $v \gets \floor{n/y}$
        
        $P \gets \dfrac{v \cdot (v-1)}{2} + \isqrt{v} \cdot \Phi_{\isqrt{v}}$
        
        \For{$t=2$ \KwTo $\isqrt{v}$}{
            $q \gets \floor{v/t}$
            
            $P \gets P - \phi_t \cdot q$
            
            \uIf{$q \leq \isqrt{n}$}{
                $P \gets P - \Phi_q$
            }
            \Else{
                $P \gets P - \Phi^\prime_{\floor{n/q}}$
            }
        }
        
        $\Phi^\prime_y \gets P$
        
        $Y \gets Y + P$
    }
    
    \KwRet $Z - X - Y$
}
\end{algorithm}


\todo\end{document}




